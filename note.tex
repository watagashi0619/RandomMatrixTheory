\documentclass{ltjsarticle}
\usepackage[
    pdfencoding=auto,
    psdextra,
    pdfusetitle,
    hidelinks,
    pdfversion=1.7,
    pdfa
]{hyperref}%ハイパーリンク
\usepackage{hyperxmp}
\hypersetup{
    pdfsubject={解析学特論Ⅱの講義に基づいてまとめたもの},
    pdfkeywords={ランダム行列},
    pdfapart=3, % PDF/A のバージョン。最新は3だが、実際は2で問題ないかも
    pdfaconformance=u, % PDF/A 内のLevel区分。b(basic), a(accesible), u(unicode)が存在
    colorlinks=true,
    linkcolor=red,
}
% カラープロファイルを埋め込んで、出力インテントの設定をしているっぽい(https://tex.stackexchange.com/a/534201)
\immediate\pdfobj stream attr{/N 3} file{sRGB.icc}
\pdfextension catalog{
    /OutputIntents [
        <<
            /Type /OutputIntent
            /S /GTS_PDFA1
            /DestOutputProfile \the\pdflastobj\space 0 R
            /OutputConditionIdentifier (sRGB)
            /Info (sRGB)
        >>
    ]
}
%\pdfvariable omitcidset=1 %デフォルト以外のフォントを読み込む場合、この一行が必要になる場合がある
%\usepackage[1.7]{bxpdfver}
\usepackage{amsthm}%定理環境
\usepackage[hiragino-pron]{luatexja-preset}
\usepackage{newtxtext,newtxmath}%times font setting
\usepackage{amsmath,amsfonts}%数式font setting
\usepackage{mathrsfs}%花文字
\usepackage{pxrubrica}%ルビ
%\usepackage{lmodern}%font setting
\usepackage{mathtools}%数式の相互参照部分にのみ式番号をふる
\mathtoolsset{showonlyrefs,showmanualtags}
\usepackage{empheq}%方程式
\usepackage{physics}%物理系
\usepackage{bm}%vector
\usepackage{mleftright}%カッコ
\usepackage{framed}%フレーム
\usepackage{color}%文字に色付け
\usepackage{ulem}%取り消し線
\usepackage{bbm}%指示関数とかはこっちで出す
%\usepackage[right]{showlabels}%数式番号を左に表示
\usepackage{booktabs}%表の上線\toprule中線\midrule下線\bottomrule
%\usepackage{url}%参考文献とかでurlをかくとき
%\usepackage{comment}%長文コメント
%\usepackage{lscape}%ページを横向きにする
\usepackage{ascmac}
\usepackage{tcolorbox}
\tcbuselibrary{theorems,skins}%数式を「デコ」る
\usepackage{tikz}
\usetikzlibrary{patterns}
\usetikzlibrary{cd}%矢印の図
\usetikzlibrary{decorations,decorations.pathreplacing}%tikzでデコる
% patternsの斜線設定
\pgfdeclarepatternformonly{south west lines}{\pgfqpoint{-0pt}{-0pt}}{\pgfqpoint{3pt}{3pt}}{\pgfqpoint{3pt}{3pt}}{
    \pgfsetlinewidth{0.4pt}
    \pgfpathmoveto{\pgfqpoint{0pt}{0pt}}
    \pgfpathlineto{\pgfqpoint{3pt}{3pt}}
    \pgfpathmoveto{\pgfqpoint{2.8pt}{-.2pt}}
    \pgfpathlineto{\pgfqpoint{3.2pt}{.2pt}}
    \pgfpathmoveto{\pgfqpoint{-.2pt}{2.8pt}}
    \pgfpathlineto{\pgfqpoint{.2pt}{3.2pt}}
    \pgfusepath{stroke}
}
%subsubsubsection定義 paragraphも用いる場合は注意
\makeatletter
    \newcommand{\subsubsubsection}{\@startsection{paragraph}{4}{\z@}%
        {1.0\Cvs \@plus.5\Cdp \@minus.2\Cdp}%
        {.1\Cvs \@plus.3\Cdp}%
        {\reset@font\sffamily\normalsize}
    }
\makeatother

\setcounter{secnumdepth}{3}%セクションの深さレベル

%physicsでrotもcurlで使う
\newcommand{\rot}{\curl}

%amsthm style1,2
%style1 連番
\newtheoremstyle{mystyle1}% Name
    {}% Space above
    {}% Space below
    {\normalfont}% Body font
    {}% Indent amount
    {\bfseries\sffamily}% Theorem head font
    {\hspace{0.5em}}% Punctuation after theorem head
    { }% Space after theorem head, ‘ ‘, or \newline
    {\thmname{#1}\thmnumber{#2}\thmnote{(#3)\\}}% Theorem head spec (can be left empty, meaning `normal')
\theoremstyle{mystyle1}
\newtheorem{dfn}{定義}[section]
\newtheorem{thm}[dfn]{定理}
\newtheorem{axi}[dfn]{公理}
\newtheorem{cor}[dfn]{系}
\newtheorem{prop}[dfn]{命題}
\newtheorem{lem}[dfn]{補題}
\newtheorem{exs}[dfn]{例題}

%style2 連番なし
\newtheoremstyle{mystyle2}% Name
    {}% Space above
    {}% Space below
    {\normalfont}% Body font
    {}% Indent amount
    {\bfseries\sffamily}% Theorem head font
    {\hspace{0.5em}}% Punctuation after theorem head
    { }% Space after theorem head, ‘ ‘, or \newline
    {\thmname{#1}\thmnote{(#3)\\}}% Theorem head spec (can be left empty, meaning `normal')
\theoremstyle{mystyle2}
\newtheorem{dfn*}{定義}
\newtheorem{thm*}{定理}
\newtheorem{prop*}{命題}
\newtheorem{ex}{例題}
\newtheorem{example}{例}
\newtheorem{qes}{問題}
\newtheorem{note*}{注意}
\newtheorem{ans}{解答}
\newtheorem{note}{注意}
\newtheorem{lem*}{補題}

\newtheoremstyle{mystyle3}% Name
    {}% Space above
    {}% Space below
    {\normalfont}% Body font
    {}% Indent amount
    {\bfseries\sffamily}% Theorem head font
    {\hspace{0.5em}}% Punctuation after theorem head
    { }% Space after theorem head, ‘ ‘, or \newline
    {\thmname{#1}\thmnumber{#2}}% Theorem head spec (can be left empty, meaning `normal')
\theoremstyle{mystyle3}
\newtheorem{qes*}{問題}

%proofの後ろに黒四角をつける
\makeatletter
\renewenvironment{proof}[1][\proofname]{\par
  \pushQED{\qed}%
  \normalfont
  \topsep6\p@\@plus6\p@ \trivlist
  \item[\hskip\labelsep{\bfseries\sffamily #1}]\ignorespaces
}{%
  \popQED\endtrivlist\@endpefalse
}
\renewcommand\proofname{証明}
\renewcommand{\qedsymbol}{\ensuremath{\blacksquare}}
\makeatother

%ansの後ろに黒四角をつける
\makeatletter
\renewenvironment{ans}[1][解答]{\par
  \pushQED{\qed}%
  \normalfont
  \topsep6\p@\@plus6\p@ \trivlist
  \item[\hskip\labelsep{\bfseries\sffamily #1}]\ignorespaces
}{%
  \popQED\endtrivlist\@endpefalse
}
\renewcommand{\labelenumi}{\ensuremath{\blacksquare}}
\makeatother

%方程式番号にセクション番号を入れる
\makeatletter
\@addtoreset{equation}{section}
\def\theequation{\thesection.\arabic{equation}}% renewcommand でもOK
\makeatother

% 図番号にセクション番号入れる
\makeatletter
\renewcommand{\thefigure}{
\arabic{figure}}
\@addtoreset{figure}{section}
\makeatother

%\newcommand\range{\operatorname{range}}
%\newcommand\sgn{\operatorname{sgn}}

%\renewcommand{\thepart}{\arabic{part}}
%\renewcommand{\thenote}{}
%\renewcommand{\thelem}{}

%\makeatletter
%\@addtoreset{section}{part}
%\makeatother

%\makeatletter
%\def\blfootnote{\xdef\@thefnmark{}\@footnotetext}
%\makeatother

%\setcounter{tocdepth}{1}%目次をsectionまでの表示にする

\renewcommand{\labelenumi}{(\arabic{enumi})}%enumerateのデフォルトを(1)などにする

%その他文字設定
\newcommand{\defLeftrightarrow}{\overset{\text{def}}{\iff}}

\title{解析学特論Ⅱ 講義ノート}
\author{}
\date{}
\begin{document}
\maketitle

\begin{dfn}[Random Matrix]
    ランダム行列$X=(x_{ij})_{i,j=1}^n$とは,各成分$x_{ij}$が確率変数であるような行列のことである.
\end{dfn}

\section{Gaussian Unitary Ensemble}

$V$は$\mathbb{R}$上の有限次元ベクトル空間であり,各要素は$r\in\mathbb{R}$として$\mathcal{N}(0,r)$に従う実数確率変数であるとする.ただし,$r=0$の場合も確率変数$0$とみなして許容する.

\begin{example}
    $g_1,\ldots,g_d\overset{\textrm{i.i.d.}}{\sim}\mathcal{N}(0,1)$としたとき,$V=\{\sum_{i=1}^d \lambda_ig_i\mid (\lambda_1,\ldots,\lambda_d)\in\mathbb{R}^d\}$はgaussian space.実際,すべてのgaussian spaceはこのように表現できる.
\end{example}

\subsection{Wick's Theorem}

$g_1,\ldots,g_n\in V$を考え,$E[g_1\cdots g_n]$の値がどうなるかを考えたい.$E[g_1\cdots g_n]$が有限であることはHölderの不等式から導ける.また,$n$が奇数の場合にこの値が0となることは容易にわかる.

\begin{dfn}[pair partition $\mathcal{P}_2(n)$]\
    \vspace{-\baselineskip}
    \begin{enumerate}
        \item $n\in\mathcal{N}$に対して$[n]=\{1,\ldots,n\}$と表記する.
        \item $[n]$のpairing $\pi$とは,$[n]$を被りのないように2つずつ選んだ数のペアに分割することである.すなわち,$\pi=\{V_1,\ldots,V_k\}$は$i,j=1,\ldots,k$(ただし$i\neq j$)に対して
              \begin{itemize}
                  \item $V_i\subset [n]$
                  \item $|V_i|=2$
                  \item $V_i\cap V_j=\emptyset$
                  \item $\bigcup_{i=1}^k V_i=[n]$
              \end{itemize}
              をみたす.なお,必然的に$k=n/2$となる.
        \item $[n]$のpair partition $\mathcal{P}_2(n)$は
              \begin{equation}
                  \mathcal{P}_2(n)=\{\pi\mid\pi\textrm{\,is\,pairing\,of\,}[n]\}
              \end{equation}
              で定義される.
    \end{enumerate}
\end{dfn}

\begin{prop}
    \begin{equation}
        \#\mathcal{P}_2(n)=\begin{cases}
            0       & (nは奇数) \\
            (n-1)!! & (nは偶数)
        \end{cases}
    \end{equation}
\end{prop}

\begin{proof}
    $\mathcal{P}_2(n)$のうちで,まず$1$とのペアを考えると,その選び方は$n-1$通りある.このペアを取り除くと,pairingを作る数は$n-2$個残っている.したがって$\#\mathcal{P}_2(n)=(n-1)\cdot\#\mathcal{P}_2(n-2)$が成り立つ.また,$\#\mathcal{P}_2(1)=0$,$\#\mathcal{P}_2(2)=1$であることと合わせれば,証明は終了する.
\end{proof}

\begin{example}
    $\mathcal{P}_2(2)=\{\{\{1,2\}\}\}$,$\mathcal{P}_2(4)=\{\{\{1,2\},\{3,4\}\},\{\{1,3\},\{2,4\}\},\{\{1,4\},\{2,3\}\}\}$.
\end{example}

\begin{dfn}
    $E_\pi [X_1,\ldots,X_n]=\prod_{(i,j)\in\pi}E[X_iX_j]$
\end{dfn}

\begin{example}
    \begin{equation}
        \begin{split}
            E[X_1X_2X_3X_4] & =E_{\{\{1,2\},\{3,4\}\}}[X_1,X_2,X_3,X_4]+E_{\{\{1,3\},\{2,4\}\}}[X_1,X_2,X_3,X_4]+E_{\{\{1,4\},\{2,3\}\}}[X_1,X_2,X_3,X_4] \\
            & =E[X_1X_2]E[X_3X_4]+E[X_1X_3]E[X_2X_4]+E[X_1X_4]E[X_2X_3]
        \end{split}
    \end{equation}
\end{example}

\begin{lem}
    $V$を$\mathbb{R}$上のベクトル空間として,$n$重線形対称関数$\varphi:V^n\to\mathbb{R}$を考える.このとき,$\varphi(x,\cdots,x)=\phi(x)$とおくと
    \begin{equation}
        \varphi(x_1,\ldots,x_n)=\frac{1}{n!}\sum_{k=1}^n\sum_{1\leq j_1<\cdots<j_k\leq n}(-1)^{n-k}\phi(x_{j_1}+\cdots+x_{j_k})
    \end{equation}
    が成り立つ.
    %\url{https://math.stackexchange.com/questions/481167/polarization-formula}
\end{lem}

\begin{note*}
    上の補題において対称とは,$\sigma\in S_n$に対して$\varphi(x_1,\ldots,x_n)=\varphi(x_{\sigma(1)},\ldots,x_{\sigma(n)})$が成り立つことである.
\end{note*}

\begin{note*}
    この等式は,偏極恒等式の一般化である.$n=2$の場合は偏極恒等式
    \begin{equation}
        \varphi(x,y)=\frac{1}{2}(\phi(x+y)-\phi(x)-\phi(y))=\frac{1}{4}(\phi(x+y)-\phi(x-y))
    \end{equation}
    となる.
\end{note*}

\begin{proof}
    \begin{equation}
        \begin{split}
            \sum_{k=1}^n\sum_{1\leq j_1<\cdots<j_k\leq n}(-1)^k\phi(x_{j_1}+\cdots+x_{j_k})
            & =\sum_{A\subset[n]}(-1)^{|A|}\phi\mleft(\sum_{a\in A}x_a\mright)                                      \\
            & =\sum_{A\subset [n]}(-1)^{|A|}\sum_{f:[n]\to A}\varphi(x_{f(1)},\ldots,x_{f(n)})                      \\
            & =\sum_{f:[n]\to[n]}\varphi(x_{f(1)},\ldots,x_{f(n)})\sum_{\mathrm{Im}f\subset A\subset [n]}(-1)^{|A|} \\
            & =\sum_{f\in S_n}\varphi(x_{f(1)},\ldots,x_{f(n)})(-1)^n                                               \\
            & =(-1)^nn!\varphi(x_1,\ldots,x_n)
        \end{split}
    \end{equation}
    となることから従う.ただし,$\mathrm{Im}f\neq [n]$のときは
    \begin{equation}
        \begin{split}
            \sum_{\mathrm{Im}f\subset A\subset [n]}(-1)^{|A|}
            & =\sum_{j=0}^{n-|\mathrm{Im}f|}(-1)^j{}_{n-|\mathrm{Im}f|}C_j \\
            & =(1+(-1))^{n-|\mathrm{Im}f|}                                 \\
            & =0
        \end{split}
    \end{equation}
    となることを利用した.
\end{proof}

\begin{lem}
    $n$重線形対称関数$\phi:V^n\to\mathbb{R}$に対して,任意の$x_1,\ldots,x_n\in V$に対して$\varphi(x_1,\ldots,x_n)=0$が成り立つことと,任意の$x\in V$に対して$\varphi(x,\ldots,x)=\phi(x)=0$が成り立つことは同値.
\end{lem}

\begin{proof}
    十分性は容易.必要性は先の補題において右辺に現れる各$\phi(x_{j_1}+\cdots+x_{j_k})=0$となるときを考えることになるため,$\varphi(x_1,\ldots,x_n)=0$が成り立つ.
\end{proof}

\begin{thm}[Wick's Theorem]
    $g_1,\ldots,g_n\in V$に対して,以下が成り立つ:
    \begin{equation}
        E[g_1\cdots g_n]=\sum_{\pi\in\mathcal{P}_2(n)}E_\pi[g_1,\ldots,g_n]
    \end{equation}
\end{thm}

\begin{proof}
    まずは$g_1=\cdots=g_n=g\in V$のときに成り立つことを示す.いま,$V$の仮定から$E[g]=0$である.さらに,$E[g^2]=1$と置いても一般性は失われない(以下の議論で$g$を$g/E[g^2]$に置き換えればよい).左辺は
    \begin{equation}
        \begin{split}
            E[g_1\cdots g_n]
            & =E[g^n]                                                                      \\
            & =\int_{-\infty}^\infty x^n\frac{1}{\sqrt{2\pi}}e^{-\frac{x^2}{2}}dx          \\
            & =(n-1)\int_{-\infty}^\infty x^{n-2}\frac{1}{\sqrt{2\pi}}e^{-\frac{x^2}{2}}dx \\
            & =\cdots                                                                      \\
            & =(n-1)!!
        \end{split}
    \end{equation}
    であり,右辺は
    \begin{equation}
        \begin{split}
            \sum_{\pi\in\mathcal{P}_2(n)}E_\pi[g,\cdots,g]
            & =(n-1)!!E[g^2]^{\frac{n}{2}} \\
            & =(n-1)!!
        \end{split}
    \end{equation}
    となることから成り立つ.いま,補題において$\varphi(g_1,\ldots,g_n)=E[g_1\cdots g_n]-\sum_{\pi\in\mathcal{P}_2(n)}E_\pi[g_1,\ldots,g_n]$と置けば,$\varphi(g,\ldots,g)=0$は確認したので$\varphi(g_1,\ldots,g_n)=0$も成り立つ.よって示された.
\end{proof}

\subsection{Gaussian Unitary Ensemble and Wiegner's Theorem}

\begin{dfn}[Gaussian Unitary Ensemble]
    Gaussian Unitary Ensembleとは,行列$H=(h_{ij})_{i,j=1}^n$であって,すべての$i,j=1,\ldots,n$に対して$h_{ij}=\overline{h}_{ji}$を満たし,$\{h_{ij}\mid i\geq j\}$は独立同分布で$\mathcal{N}(0,1/n)$に従うようなもののことである.$H$のガウス測度(すなわち,$H$のすべての要素の同時分布)は
    \begin{equation}
        d\mu(H)=\frac{1}{Z}e^{-\frac{n}{2}\mathrm{tr}H^2}dH
    \end{equation}
    となる.ただし$Z=2^{\frac{n}{2}}\pi^{\frac{1}{2}n^2}$,$dH=\prod_{i\geq j}dh_{ij}$.
\end{dfn}

\begin{example}
    ランダム行列$X$は$\mathrm{GUE}(n)$であるとする.このとき,$E[X_{ij}X_{kl}]=\delta_{il}\delta_{jk}/n$が成り立つ.また,Wickの定理より$E[X_{i_1j_1}\cdots X_{i_kj_k}]=\sum_{\pi\in\mathcal{P}_2(k)}n^{-k/2}\delta_{\pi_{ij}}$.
\end{example}

\begin{dfn}[互換の数]
    $k$次の対称群の元$\sigma\in S_k$に対して$|\sigma|$を互換の数として定める.
\end{dfn}

\begin{dfn}[サイクルの数]
    $k$次の対称群の元$\sigma\in S_k$に対して$\#(\sigma)$を,巡回置換の積に分解したときのサイクルの数として定める($|\sigma|$ とは異なることに注意).
\end{dfn}

\begin{prop}
    \
    \begin{enumerate}
        \item $\sigma,\tau \in S_k$に対して,対応$(\sigma,\tau)\mapsto |\sigma\tau^{-1}|$を考えると,これは$S_k$の距離となる.すなわち,この対応を$d$で表すと以下の3つが成り立つ:
              \begin{itemize}
                  \item $d(\sigma,\tau)=0\leftrightarrow \tau=\sigma$
                  \item $d(\sigma_1,\sigma_2)+d(\sigma_2,\sigma_3)\geq d(\sigma_1,\sigma_3)$
                  \item $d(\sigma_1,\sigma_2)=d(\sigma_2,\sigma_1)$
              \end{itemize}
        \item $\sigma\in S_k$について,$|\sigma|+\#(\sigma)=k$が成り立つ.
    \end{enumerate}
\end{prop}

\begin{example}
    以下のいずれの場合でも$|\sigma|+\#(\sigma)=k$が成立していることが確認できる.
    \begin{itemize}
        \item $\sigma=e\in S_k$を考えると,$|\sigma|=0$.$\sigma=(1)(2)\cdots(k)$より$\#(e)=k$.
        \item $\sigma=(1\,2)\in S_k$を考えると,$|\sigma|=1$,$\sigma=(1\,2)(3)\cdots(k)$より$\#(\sigma)=k-1$.
        \item $\sigma=(1\,2\,\cdots\,k)\in S_k$を考えると,$|\sigma|=|(1\,2)\cdots(k-1\,k)|=k-1$.$\#(\sigma)=1$.
    \end{itemize}
\end{example}

\begin{dfn}[normalized trace]
    $n$次元正方行列$X$に対して$\mathrm{tr}X=\frac{1}{n}\mathrm{Tr}X$と定める.
\end{dfn}

\begin{thm}
    $X$を$\mathrm{GUE}(n)$とする.このとき,$\gamma=(1\,2\,\cdots\,k)\in S_k$として以下が成り立つ:
    \begin{equation}
        E[\mathrm{tr}X^k]=\sum_{\pi\in\mathcal{P}_2(k)}n^{-1-\frac{k}{2}+\#(\gamma\pi)}
    \end{equation}
\end{thm}

\begin{proof}
    \begin{equation}
        \mathrm{tr}X^k=\frac{1}{n}\sum_{1\leq i_1,i_2,\ldots,i_k\leq n}X_{i_1i_2}X_{i_2i_3}\cdots X_{i_ki_1}
    \end{equation}
    より
    \begin{equation}
        E[\mathrm{tr}X^k]=\frac{1}{n}\sum_{1\leq i_1,i_2,\ldots,i_k\leq n}E[X_{i_1i_2}X_{i_2i_3}\cdots X_{i_ki_1}]
    \end{equation}
    ここで,Wickの定理より
    \begin{equation}
        \begin{split}
            E[X_{i_1i_2}X_{i_2i_3}\cdots X_{i_ki_1}]
            & =\sum_{\pi\in\mathcal{P}_2(k)}E_\pi[X_{i_1i_2},X_{i_2i_3},\ldots, X_{i_ki_1}]    \\
            & =\sum_{\pi\in\mathcal{P}_2(k)}\prod_{(a,b)\in\pi}E[X_{i_ai_{a+1}}X_{i_bi_{b+1}}]
        \end{split}
    \end{equation}
    ただし$i_{k+1}=i_1$とする.ここで
    \begin{equation}
        E[X_{i_ai_{a+1}}X_{i_bi_{b+1}}]=\frac{1}{n}\delta_{i_ai_{a+1}}\delta_{i_bi_{b+1}}
    \end{equation}
    であることと,$(a,b)\in\pi$は$\pi(a)=b,\pi(b)=a$であるということなので
    \begin{equation}
        \prod_{(a,b)\in\pi}\delta_{i_ai_{a+1}}\delta_{i_bi_{b+1}}=\prod_{a=1}^k\delta_{i_ai_{\pi(a)+1}}
    \end{equation}
    となる.ここでshift permutation $\gamma\in S_k$(すなわち$\gamma=(1\,2\,\cdots\,k),\gamma(a)=a+1\mod k$)を定めると結局
    \begin{equation}
        \begin{split}
            E[\mathrm{tr}X^k]=\frac{1}{n}\frac{1}{n^{k/2}}\sum_{\pi\in\mathcal{P}_2(k)}\sum_{1\leq i_1,i_2,\ldots,i_k\leq n}\prod_{a=1}^k\delta_{i_ai_{\gamma\pi(a)}}
        \end{split}
    \end{equation}
    となる($\mathcal{P}_2(k)=k/2$に注意).ここで$\prod_{a=1}^k\delta_{i_ai_{\gamma\pi(a)}}\neq 0$であるのは$\gamma\pi$のサイクル上で一定となっていることである.したがって%どういうこと?
    \begin{equation}
        E[\mathrm{tr}X^k]=\frac{1}{n^{k/2+1}}\sum_{\pi\in\mathcal{P}_2(k)}n^{\#(\gamma\pi)}
    \end{equation}
\end{proof}

\begin{note*}
    証明中の次の部分に対してグラフを利用した解釈を与えよう.
    \begin{equation}
        \begin{split}
            E[X_{i_1i_2}X_{i_2i_3}\cdots X_{i_ki_1}]
            & =\sum_{\pi\in\mathcal{P}_2(k)}E_\pi[X_{i_1i_2},X_{i_2i_3},\ldots, X_{i_ki_1}]    \\
            & =\sum_{\pi\in\mathcal{P}_2(k)}\prod_{(a,b)\in\pi}E[X_{i_ai_{a+1}}X_{i_bi_{b+1}}]
        \end{split}
    \end{equation}
    まず,$i_1\,i_2,\ldots,i_k$ の列について,この列の中に現れる重複なしindexの数を $w$ とする.もし,このindexの列の中に1つしか現れないindex $i_l$があれば,各確率変数は独立であるから,あるindex $i_m$に対して$E[X_{i_li_m}]=E[X_{i_li_m}]=0$が成り立つため,$E[X_{i_1i_2}X_{i_2i_3}\cdots X_{i_ki_1}]=0$となることがわかる.したがって,この計算の値に寄与するとき,各indexは$i_1\,i_2,\ldots,i_k$の列の中に少なくとも2回ずつ現れる.したがって $w\leq k/2+1$ であることがわかる.このとき,この $w$ を与えるような列$i_1\,i_2,\ldots,i_k$の与え方は $n(n-1)\cdots(n-w+1)\leq n^w\leq n^{k/2+1}$となる.したがって$E[X_{i_1i_2}X_{i_2i_3}\cdots X_{i_ki_1}]=O(n^{k/2+1})$であることがわかる.最後の段階には$n^{-k/2-1}$が掛けられるので,寄与として残る可能性があるのは $n^{k/2+1}$ のオーダーについてのみである.したがって,特に $w=k/2+1$ の場合を考える.いま,列$i_1\,i_2,\ldots,i_k$に対して,有向グラフ$G=(V,E)$ を $V$ は $i_1,\ldots,i_k$ から重複なくとった集合, $E=\{(i_1,i_2),\ldots,(i_k,i_1)\}$ とする.列$i_1\,i_2,\ldots,i_k$の長さは$k$なので,辺 の数を重複なく数えると $k/2$ 個となる.すなわち,各辺はちょうど2回通過する.言い換えれば任意の辺 $(i_j,i_{j+1})$ に対して $i_j=i_{l+1},i_{j+1}=i_l$ をみたすような辺 $(i_l,i_{l+1})$ が存在する.この性質をみたす任意のパスがnon-crossingなものとなっている.
\end{note*}

上の定理の別の表記を与える.

\begin{thm}
    $X$を$\mathrm{GUE}(n)$とする.このとき,$\gamma=(1\,2\,\cdots\,k)\in S_k$として
    \begin{equation}
        E[\mathrm{tr}X^k]=\sum_{\pi\in\mathcal{P}_2(k)}n^{|\gamma|-|\pi|-|\gamma\pi|}
    \end{equation}
    が成り立つ.
\end{thm}

\begin{proof}
    $|\gamma|=k-1$,$|\pi|=k/2$,$|\gamma\pi|=k-\#(\gamma\pi)$であることから確認できる.
\end{proof}

\begin{example}
    $X$が$\mathrm{GUE}(n)$であるとき,$E[\mathrm{tr}X^4]=2+\frac{1}{n}$
\end{example}

\begin{proof}
    定義とWickの定理から計算すれば
    \begin{equation}
        \begin{split}
            E[\mathrm{tr}X^4]
            & =\frac{1}{n^3}\sum_{i,j,k,l=1}^nE[X_{ij}X_{jk}X_{kl}X_{li}]                               \\
            & =\frac{1}{n^3}\sum_{i,j,k,l=1}^n(E_{\pi_1}+E_{\pi_2}+E_{\pi_3})[X_{ij}X_{jk}X_{kl}X_{li}]
        \end{split}
    \end{equation}

    各項について
    \begin{equation}
        \begin{split}
            \sum_{i,j,k,l=1}^n E_{\pi_2}[X_{ij},X_{jk},X_{kl},X_{li}]
            & =\sum_{i,j,k,l=1}^n E[X_{ij}X_{jk}]E[X_{kl}X_{li}]               \\
            & =\sum_{i,j,k,l=1}^n \delta_{ik}\delta_{jj}\delta_{ki}\delta_{ll} \\
            & =n^3
        \end{split}
    \end{equation}
    \begin{equation}
        \begin{split}
            \sum_{i,j,k,l=1}^n E_{\pi_1}[X_{ij},X_{jk},X_{kl},X_{li}]
            & =\sum_{i,j,k,l=1}^n E[X_{ij}X_{kl}]E[X_{jk}X_{li}]               \\
            & =\sum_{i,j,k,l=1}^n \delta_{il}\delta_{jk}\delta_{ji}\delta_{kl} \\
            & =n
        \end{split}
    \end{equation}
    \begin{equation}
        \begin{split}
            \sum_{i,j,k,l=1}^n E_{\pi_3}[X_{ij},X_{jk},X_{kl},X_{li}]
            & =\sum_{i,j,k,l=1}^n E[X_{ij}X_{li}]E[X_{jk}X_{kl}]               \\
            & =\sum_{i,j,k,l=1}^n \delta_{ii}\delta_{jl}\delta_{jl}\delta_{kk} \\
            & =n^3
        \end{split}
    \end{equation}
    となることから従う.

    また,先の定理を使えば,とられる和の中身が以下の表のようになることからもわかる.
    \begin{table}[hbtp]
        \centering
        \begin{tabular}{cccc}
            \hline
            $\pi$          & $\gamma\pi$    & $\#(\gamma\pi)-3$ & 貢献度   \\
            \hline \hline
            $(1\,2)(3\,4)$ & $(1\,3)(2)(4)$ & $0$               & $n^0=1$  \\
            $(1\,3)(2\,4)$ & $(1\,4\,3\,2)$ & $-2$              & $n^{-2}$ \\
            $(1\,4)(2\,3)$ & $(1)(2\,4)(3)$ & $0$               & $n^0=1$  \\
            \hline
        \end{tabular}
    \end{table}
\end{proof}

\begin{dfn}[non-crossing]
    $\pi\in\mathcal{P}_2(m)$がnon-crossingとは,すべての$\pi$の要素$(i,k)$と$(j,l)$について,$i<j<k<l$とはならないことである.non-crossingな集合のことを
    \begin{equation}
        \mathcal{NC}_2(m)=\{\pi\in\mathcal{P}_2(m)\mid\pi\,\textrm{ is\, non-crossing}\}
    \end{equation}
    と表記する.
\end{dfn}

\begin{thm}[Biane's lemma]
    $\pi\in\mathcal{P}_2(k)\subset S_k$とし,$\gamma=(1\,2\,\cdots\,k)\in S_k$とする.
    \begin{enumerate}
        \item $\#(\gamma\pi)\leq\frac{k}{2}+1$.
        \item $\#(\gamma\pi)=\frac{k}{2}+1$であることは$\pi\in\mathcal{NC}_2(k)$であるための必要十分条件である.
    \end{enumerate}
    \begin{comment}
    別の書き方をすれば
    \begin{enumerate}
        \item $|\gamma|\leq |\pi|+|\gamma\pi|$.
        \item $|\gamma|=|\pi|+|\gamma\pi|$であることは,$\pi\in\mathcal{NC}_2(k)$であるための必要十分条件である.
    \end{enumerate}
    \end{comment}
\end{thm}

\begin{proof}
    $\tau=(i\,i+1)\in\pi$であったとすれば,$\gamma\pi(i+1)=i+1$,$\gamma\pi(i)=i+2$となる.したがって,$\gamma\pi$はサイクル$(i+1)$と$(\cdots\,i\,i+2\,\cdots)$を含む.このとき$i,i+1$を取り除く操作を行う.その場合の$\pi$と$\gamma$を改めて$\pi_2$と$\gamma_2$として考え,再び同様にして取り除く操作を繰り返すことを考える.

    もし$\pi\in\mathcal{NC}_2(k)$であれば,上の操作は$\pi_{k/2-1}=(1\,2)$,$\gamma_{k/2-1}=(1\,2)$,$\gamma_{k/2-1}\pi_{k/2-1}=(1)(2)$,$\#(\gamma_{k/2-1}\pi_{k/2-1})=2$となるまで繰り返すことができる.これは最初の状態から上の操作を$k/2-1$回繰り返すことで得られる状態である.よって$\#(\gamma\pi)=(k/2-1)+\#(\gamma_{k/2-1}\pi_{k/2-1})=k/2+1$となる.

    一方で,$\pi\notin\mathcal{NC}_2(k)$であれば,上の操作を繰り返し適用し,これ以上操作ができない状態に到達したとして,そのときの$\pi_i\gamma_i$を考えれば,$\pi_i\gamma_i(j)=j$となるような$j$が存在しない.すなわち,$\pi_i\gamma_i$に存在するサイクルは必ず2つ以上の要素によって構成されていることになる.したがって,始めの状態に戻して考えることで$\#(\gamma\pi)\leq k/2<k/2+1$であることがわかる.
\end{proof}


\begin{lem}\label{lem:1-16}
    \begin{equation}
        \#\mathcal{NC}_2(n)=
        \begin{cases}
            0                                      & (nは奇数) \\
            \mathrm{Cat}\mleft(\frac{n}{2}\mright) & (nは偶数)
        \end{cases}
    \end{equation}
    ただし$\mathrm{Cat}$はカタラン数であり,$\mathrm{Cat}(m)=\frac{(2m)!}{m!(m+1)!}$をみたす.
\end{lem}

\begin{proof}
    いま,$k\in[2(n+1)]$を任意に選んで,$k$と$2(n+1)$を結ぶことを考えると,$[k-1]$と$[2n+1]/[k]\cong [2n-(k-1)]$の2つの集合についてnon-crossing pairを考えればよいことになるので,$n$が奇数のときは$\#\mathcal{NC}_2(n)=0$であることに注意して,全ての$k$について和をとり
    \begin{equation}
        \begin{split}
            \#\mathcal{NC}_2(2(n+1)) & =\sum_{k=1}^{2n+1}\#\mathcal{NC}_2(k-1)\#\mathcal{NC}_2(2n-(k-1)) \\
            & =\sum_{k=0}^{n}\#\mathcal{NC}_2(2k)\#\mathcal{NC}_2(2(n-k))
        \end{split}
    \end{equation}
    となる.ここで$c_{n}=\#\mathcal{NC}_2(2n)$とおくことで
    \begin{equation}
        c_{n+1}=\sum_{k=0}^n c_kc_{n-k}
    \end{equation}
    となる.この$c_n=\mathrm{Cat}(n)$である.実際,この数列の母関数$f(x)$を考えると
    \begin{equation}
        f(x)=\sum_{k=0}^\infty c_kx^k
    \end{equation}
    となるが,
    \begin{equation}
        \begin{split}
            f(x)^2 & =\sum_{n=0}^\infty\sum_{k=0}^n c_kc_{n-k} x^n \\
            & =\sum_{n=0}^\infty c_{n+1}x^n                 \\
            & =\frac{1}{x}(f(x)-1)
        \end{split}
    \end{equation}
    であることより
    \begin{equation}
        f(x)^2x-f(x)+1=0
    \end{equation}
    を解いて(ただし$f(0)=c_0=1$で連続となるように符号をとる)
    \begin{equation}
        \begin{split}
            f(x) & =\frac{1-\sqrt{1-4x}}{2x}                    \\
            & =\sum_{n=1}^\infty \frac{(2n)!}{(n+1)!n!}x^n
        \end{split}
    \end{equation}
    係数を比較することで$c_n=\frac{(2n)!}{(n+1)!n!}$を得る.

\end{proof}

$\lim_{n\to\infty}E[\mathrm{tr} X^k]$の値を考えたい.このとき,$\#(\gamma\pi)\leq \frac{k}{2}+1$であることから,$\pi\in\mathcal{P}_2(k)$が$\#(\gamma\pi)= \frac{k}{2}+1$をみたす場合だけ考えれば十分.なお,幾何的には$\pi\in\mathcal{P}_2(k)$が先の命題で定めた距離$d$に関する測地線上にあることを意味する.

\begin{thm}\label{thm:1-17}
    $X$は$\mathrm{GUE}(n)$であるとする.このとき,以下が成り立つ:
    \begin{equation}
        \lim_{n\to\infty}E[\mathrm{tr} X^k]=\#\mathcal{NC}_2(k)
    \end{equation}
\end{thm}

\begin{dfn}[semicircular distribution]
    \begin{equation}
        ds(x)=\frac{1}{2\pi}\sqrt{4-x^2}\mathbbm{1}_{\{|x|\leq2\}}
    \end{equation}
\end{dfn}

\begin{example}[Haar measure]
    $U\in\mathrm{SU}(2)$をとると,$\mathrm{Tr}U$は$s$による分布をもつ.
\end{example}

\begin{lem}\label{lem:1-19}
    \begin{equation}
        \int x^k ds(x)=\begin{cases}
            0                         & (kは奇数) \\
            \mathrm{Cat}(\frac{k}{2}) & (kは偶数)
        \end{cases}
    \end{equation}
\end{lem}

\begin{proof}
    部分積分により示せる.
\end{proof}

以上より,次が成り立つ.

\begin{thm}\label{thm:1-20}
    $X^{(n)}$を$\mathrm{GUE}(n)$とする.任意の実数係数多項式$p\in\mathrm{R}[x]$に対して
    \begin{equation}
        \lim_{n\to\infty}E[\mathrm{tr}p(X^{(n)})]=\int p(x)ds(x)
    \end{equation}
    が成り立つ.
\end{thm}

\begin{proof}
    補題\ref{lem:1-16}と定理\ref{thm:1-17}と補題\ref{lem:1-19}から,任意の実数係数多項式$p\in\mathrm{R}[x]$に対して
    \begin{equation}
        \lim_{n\to\infty}E[\mathrm{tr}(X^{(n)})^k]=\int x^kds(x)
    \end{equation}
    が成り立つことと,トレースの線形性から従う.
\end{proof}


\begin{dfn}
    エルミート行列$Z^{(n)}$の標準化された固有値計数測度を,$Z^{(n)}$の固有値を$\lambda_1(Z^{(n)})\geq\cdots\geq\lambda_n(Z^{(n)})$と表記して
    \begin{equation}
        \mu_{Z^{(n)}}=\frac{1}{n}\sum_{i=1}^n\delta_{\lambda_i(Z^{(n)})}
    \end{equation}
    で定める.また
    \begin{equation}
        \mu_{Z^{(n)}}([a,b])=\frac{1}{n}\#\{i\mid\lambda_i(Z)\in[a,b]\}
    \end{equation}
    である.
\end{dfn}

\begin{lem}
    エルミート行列$Z^{(n)}$に対して,任意の実数係数多項式$p\in\mathrm{R}[x]$に対して
    \begin{equation}
        \mathrm{tr}p(Z^{(n)})=\int p(t)d\mu_{Z^{(n)}}(t)
    \end{equation}
    が成り立つ.
\end{lem}

\begin{proof}
    \begin{equation}
        \begin{split}
            \mathrm{tr}p(Z^{(n)}) & =\frac{1}{n}\sum_{i=1}^np(\lambda_i(Z^{(n)}))                             \\
            & =\int_\mathbb{R}p(x)\frac{1}{n}\sum_{i=1}^n\delta_{\lambda_i(Z^{(n)})}(x)
        \end{split}
    \end{equation}
\end{proof}

$Z^{(n)}$が決定論的なら$\mu_{Z^{(n)}}$は決定論的,$Z^{(n)}$がランダム行列ならばランダム確率測度という.$\mu_{X^{(n)}}$がランダム確率測度ならば特に$E\mu_{X^{(n)}}$がwell-definedとなる.これは任意の連続関数$f$に対して
\begin{equation}
    E\int f(x)d\mu_{X^{(n)}}(x)=\int f(x)dE\mu_{X^{(n)}}(x)
\end{equation}
である.これを用いると次が示せる:

\begin{thm}\label{thm:1-24}
    $X^{(n)}$を$\mathrm{GUE}(n)$とする.任意の実数係数多項式$p\in\mathbb{R}[x]$に対して
    \begin{equation}
        \int p(t)dE\mu_{X^{(n)}}(t)\overset{n\to\infty}{\longrightarrow}\int p(t)ds(t)
    \end{equation}
    が成り立つ.
\end{thm}

\begin{proof}
    定理\ref{thm:1-20}を$\mu_{X^{(n)}}$ を用いて書き換えたものである.
\end{proof}

\begin{thm}[Wiegner's (semicircular) theorem]\label{thm:1-25}
    $X^{(n)}$を$\mathrm{GUE}(n)$とする.任意の有界連続関数$f:\mathbb{R}\to\mathbb{R}$に対して
    \begin{equation}
        \int f(t)dE\mu_{X^{(n)}}(t)\overset{n\to\infty}{\longrightarrow}\int f(t)ds(t)
    \end{equation}
    が成り立つ.言い換えると,
    \begin{equation}
        E\mu_{X^{(n)}}\overset{n\to\infty}{\longrightarrow} s\quad(\textrm{in\, weak})
    \end{equation}
    が成り立つ.
\end{thm}

\begin{proof}
    後の定理\ref{thm:1.28}で正確に証明する.
\end{proof}

\begin{note*}
    この定理には,複数のバリエーションがある.例えば,次のようなものもある.
\end{note*}
\begin{thm}[Wiegner's (semicircular) theorem: another variant]
    $X^{(n)}=(x_{ij}^{(n)})$ を次の条件をみたす実数ランダム行列とする(以下の条件をみたす行列をWiegner Matrixという):
    \begin{itemize}
        \item $x_{ij}^{(n)}=x_{ji}^{(n)}$.
        \item $(x_{ij}^{(n)})_{i\geq j}$ は独立.
        \item $E[x_{ij}^{(n)}]=0$.
        \item $E[(x_{ij}^{(n)})^2]=1/n$.
        \item すべてのモーメントは有限.
    \end{itemize}
    このとき,
    \begin{equation}
        E\mu_{X^{(n)}}\overset{n\to\infty}{\longrightarrow} s\quad(\textrm{in\, weak})
    \end{equation}
    が成り立つ.
\end{thm}

ここまでの収束は期待値の意味での収束だが,概収束の意味での収束も示すことができる(宿題).まずは多項式に対して示し,その後に有界連続関数に対して示す.

\begin{thm}\label{thm:1-26}
    $X^{(n)}$をWignerランダム行列とする.任意の実数係数多項式$p\in\mathbb{R}[x]$に対して
    \begin{equation}
        \int p(t)d\mu_{X^{(n)}}(t)\overset{n\to\infty}{\longrightarrow}\int p(t)ds(t)
    \end{equation}
    が成り立つ.
\end{thm}

\begin{proof}
    期待値の意味での収束はいえてるので,分散についてみる.
    \begin{equation}
        \begin{split}
            V[\tr(X^{(n)})^k]
            & =E[(\tr(X^{(n)})^k)^2]-(E[\tr(X^{(n)})^k])^2                                                                                                                                                               \\
            & =\frac{1}{n^{k+2}}\sum_{1\leq i_1,\ldots,i_k\leq n}\sum_{1\leq j_1,\ldots,j_k\leq n}E[X_{i_1i_2}\cdots X_{i_ki_1}X_{j_1j_2}\cdots X_{j_kj_1}]-E[X_{i_1i_2}\cdots X_{i_ki_1}]E[X_{j_1j_2}\cdots X_{j_kj_1}]
        \end{split}
    \end{equation}
    第1項の列 $i_1,\ldots,i_k,j_1,\ldots,j_k$ に対してグラフ的な解釈を考える.以前の注意でも見た通り,この列の中で重複なしindexの数を$w$とすると,$w\leq k+1$である.もし,$i_1,\ldots,i_k$ の部分と $j_1,\ldots,j_k$ の部分が完全に独立していれば,第2項と相殺されるので,グラフにおいて $i_1,\ldots,i_k$ の部分と $j_1,\ldots,j_k$ の部分は,何らかの辺で繋がっている場合について考えることになる.特に $w=k+1$ の場合を考えると,この場合は存在しないことが次のように示せる.

    このとき,グラフ $G=(V,E)$ を $V$ として $i_1,\ldots,i_k,j_1,\ldots,j_k$ から重複なく作る集合,$E=\{(i_1,i_2),\ldots,(i_k,i_1),(j_1,j_2),\ldots,(j_k,j_1)\}$としたとき $|V|=k+1,|E|=k$ となり,木となっている.同時に,$G_i=(V_i,E_i)$ を $V_i$ として $i_1,\ldots,i_k$ から重複なく作る集合,$E=\{(i_1,i_2),\ldots,(i_k,i_1)\}$ とする.このとき,列 $i_1,\ldots,i_k,j_1,\ldots,j_k$ が1つのパスになっており,無向グラフとみた場合には同一辺がちょうど2つずつある.言い換えれば,列 $i_1,\ldots,i_k,j_1,\ldots,j_k$ を 列$s_1,\ldots,s_{2k}$ に対応させたときに,任意の辺 $(s_l,s_{l+1})$ に対して $s_l=s_{m+1},s_{l+1}=s_m$ をみたすような辺 $(s_m,s_{m+1})$ が存在する.このとき,何らかの辺で $i$ と $j$ が繋がっているはずなので,$(i_l,i_{l+1})$ に対して $i_l=j_{m+1},i_{l+1}=j_m$ をみたすような辺 $(j_m,j_{m+1})$ が存在するはずである.しかし, $G_i,G_j$ 内のみでも2つの組となる辺が存在していなければならないので矛盾である.したがって $w\leq k$ であることがわかる.

    このことを認めれば,いま,各モーメントは有界であり,和をとったものは $O(n^k)$ となる.したがって $V[\tr(X^{(n)})^k]=O(1/n^2)$ であることがわかる.

    さて,マルコフの不等式を利用すると,任意の $\varepsilon>0$ に対して
    \begin{equation}
        \begin{split}
            P\mleft(\mleft|\tr(X^{(n)})^k-E[\tr(X^{(n)})^k]\mright|>\varepsilon\mright)
            & =P\mleft(\mleft|\tr(X^{(n)})^k-E[\tr(X^{(n)})^k]\mright|^2>\varepsilon^2\mright) \\
            & \leq\frac{V[\tr(X^{(n)})^k]}{\varepsilon^2}                                      \\
            & \leq\frac{C}{n^2}\quad(\exists C>0)
        \end{split}
    \end{equation}
    であることがわかる.よって
    \begin{equation}
        \sum_{n=1}^\infty P\mleft(\mleft|\tr(X^{(n)})^k-E[\tr(X^{(n)})^k]\mright|>\varepsilon\mright)\leq C\sum_{n=1}^\infty\frac{1}{n^2}<\infty
    \end{equation}
    である.したがって,Borel-Cantelliの補題より
    \begin{equation}
        P\mleft(\mleft|\tr(X^{(n)})^k-E[\tr(X^{(n)})^k]\mright|>\varepsilon\,\,\mathrm{i.o.}\mright)=0
    \end{equation}
    したがって
    \begin{equation}
        \lim_{n\to\infty}\mleft|\tr(X^{(n)})^k-E[\tr(X^{(n)})^k]\mright|=0\quad(\mathrm{a.s.})
    \end{equation}
\end{proof}

\begin{thm}[Wiegner's (semicircular) theorem: a.s. version]\label{thm:1.28}
    $X^{(n)}$をWignerランダム行列とする.任意の有界連続関数$f:\mathbb{R}\to\mathbb{R}$に対して
    \begin{equation}
        \int f(t)d\mu_{X}(t)\overset{n\to\infty}{\longrightarrow}\int f(t)ds(t)
    \end{equation}
    が成り立つ.
\end{thm}

\begin{proof}
    定理\ref{thm:1-26}を認めることにする.また,$X^{(n)}$ は対角化されていると考えても一般性を失わない.$\mu_{X^{(n)}}=\frac{1}{n}\sum_{i=1}^n\delta_{\lambda_i^{(n)}}$とすると
    \begin{equation}
        \int f d\mu_{X^{(n)}}=\frac{1}{n}\sum_{i=1}^nf(\lambda_i^{(n)})=\mathrm{tr}f(X^{(n)})
    \end{equation}
    ワイエルシュトラスの近似定理より,任意の $\varepsilon>0$ に対して,ある多項式 $ p_\varepsilon$ が存在して $\sup_{|t|\leq 3}|f(t)-p_\varepsilon(t)|<\varepsilon/2$ とできる.いま,
    \begin{equation}
        \begin{split}
            \mleft|\int f d\mu_{X^{(n)}}-\int f ds\mright|
            & \leq\mleft|\int f d\mu_{X^{(n)}}-\int p_\varepsilon d\mu_{X^{(n)}}\mright|+\mleft|\int p_\varepsilon d\mu_{X^{(n)}}-\int p_\varepsilon ds\mright|+\mleft|\int p_\varepsilon ds-\int f ds\mright| \\
            & \leq\int_{|t|\leq 3}|f(t)-p_\varepsilon(t)|d\mu_{X^{(n)}}+\int_{|t|> 3}|f(t)-p_\varepsilon(t)|d\mu_{X^{(n)}}                                                                                     \\
            & \qquad+\mleft|\int p_\varepsilon d\mu_{X^{(n)}}-\int p_\varepsilon ds\mright|+\mleft|\int p_\varepsilon ds-\int f ds\mright|
        \end{split}
    \end{equation}
    \begin{itemize}
        \item $[-3,3]$ 上で $|f(x)-p_\varepsilon(x)|<\varepsilon/2$ であり, $\mu_{X^{(n)}}$ は確率測度であるから
              \begin{equation}
                  \int_{|t|\leq 3} |f(t) - p_\varepsilon(t)| d\mu_{X^{(n)}}(t)<\frac{\varepsilon}{2}
              \end{equation}
        \item $f$ は有界なので $|f(t)-p_\varepsilon(t)|\leq\|f\|_\infty +|p_\varepsilon(t)|$ であり, $|t|>3$ では $p_\varepsilon$ の次数を $k$ とおくと,ある定数 $c>0$ が存在して $\|f\|_\infty+|p_\varepsilon(t)|\leq c|t|^k$ とできる.したがって
              \begin{equation}
                  \begin{split}
                      \int_{|t|>3}|f(t)-p_\varepsilon(t)|d\mu_{X^{(n)}}(t) & \leq\int_{|t|>3}c|t|^kd\mu_{X^{(n)}}(t)                                                                          \\
                      & \leq \frac{c}{3^{k+2l}}\int_{|t|>3}|t|^{2(k+l)}d\mu_{X^{(n)}}(t)\quad (k=2(k+l)-(k+2l),\forall l\in\mathbb{Z}_+)
                  \end{split}
              \end{equation}
              また,定理\ref{thm:1-26}を用いて
              \begin{equation}
                  \begin{split}
                      \lim_{n\to\infty}\int |t|^{2(k+l)}d\mu_{X^{(n)}}(t)
                      & =\int |t|^{2(k+l)}ds(t)\quad(\mathrm{a.s.})             \\
                      & =\int_{-2}^2t^{2(k+l)}\frac{1}{2\pi}\sqrt{4-t^2}dt      \\
                      & \leq  2^{2(k+l)}\int_{-2}^2\frac{1}{2\pi}\sqrt{4-t^2}dt \\
                      & =2^{2(k+l)}
                  \end{split}
              \end{equation}
              したがって
              \begin{equation}
                  \limsup_{n\to\infty}\int |f(t)-p_\varepsilon(t)|d\mu_{X^{(n)}}\leq \frac{c}{3^{k+2l}}\cdot 2^{2(k+l)}=c\mleft(\frac{4}{3}\mright)\mleft(\frac{2}{3}\mright)^{2l}\to 0\quad (l\to\infty)
              \end{equation}
              よって
              \begin{equation}
                  \int |f(t)-p_\varepsilon(t)|d\mu_{X^{(n)}}\to 0\quad (l\to\infty)
              \end{equation}
        \item 定理\ref{thm:1-26}を用いると
              \begin{equation}
                  \mleft|\int p_\varepsilon d\mu_{X^{(n)}}-\int p_\varepsilon ds\mright|\to 0\quad (n\to\infty)
              \end{equation}
        \item $[-3,3]$ 上で$|p_\varepsilon(t)-f(t)|<\varepsilon/2$ であり,これは $s$ の台である $[-2,2]$ を含むので
              \begin{equation}
                  \mleft|\int p_\varepsilon ds - \int f ds\mright|\leq \int |p_\varepsilon - f|ds < \frac{\varepsilon}{2}
              \end{equation}
    \end{itemize}
    以上を合わせると
    \begin{equation}
        \limsup_{n\to\infty}\mleft|\int f d\mu_{X^{(n)}}-\int f ds\mright|\leq\varepsilon
    \end{equation}
    したがって
    \begin{equation}
        P\mleft(\limsup_{n\to\infty}\mleft|\int f d\mu_{X^{(n)}}-\int f ds\mright|>\varepsilon\mright)=0
    \end{equation}
    このことは
    \begin{equation}
        \int f d\mu_{X^{(n)}}=\mathrm{tr}(f(X^{(n)}))\overset{n\to\infty}{\longrightarrow}\int f ds\quad(\textrm{a.s.})
    \end{equation}
    を意味する.
\end{proof}

発展的な内容として,GUEの行列の積がどのようになるか考えてみる.すなわち,$X_1^{(n)},\ldots,X_d^{(n)}$ を $\mathrm{GUE}(n)$ としたとき, $E[\mathrm{tr}X^{(n)}_{i_1}\cdots X^{(n)}_{i_n}]$ $i_1,\ldots,i_k\in\{1,\ldots, d\}^k$がどのようになるか?という問題である.これに対しては,次の結果が知られている:

\begin{thm}
    \begin{equation}
        E[\mathrm{tr}X^{(n)}_{i_1}\cdots X^{(n)}_{i_k}]=\sum_{\pi\in\mathcal{P}_2(k);\pi\, \textrm{is\, compatible\, with}\, \{i_1,\ldots,i_n\}} n^{|\gamma|-|\pi|-|\gamma\pi|}
    \end{equation}
\end{thm}

これを用いると,非常に一般的な結果を示すことができる.

\begin{thm}
    $p$ を非可換な $d$ 個の変数 $x_1,\ldots,x_d$ をもつNC多項式とし,$p^{(n)}$ を変数を確率変数で置き換えて得られるランダム行列とする.このとき,$p$ に依存する確率測度 $\mu_p$ が存在して
    \begin{equation}
        E\mu_{p^{(n)}}\overset{n\to\infty}{\longrightarrow}\mu_p
    \end{equation}
\end{thm}

\begin{note*}
    例えば $p=x_1x_2+x_2x_1$ なら $p^{(n)}=X_1^{(n)}X_2^{(n)}+X_2^{(n)}X_1^{(n)}$.
\end{note*}

\begin{note*}
    Wiegner's Theoremは $p=x_1$ の場合で,このとき $\mu_p=s$ となる.しかし,一般の $p$ に対して $\mu_p$ の分布を調べることは困難である.これは現在の自由確率論の目標である.ただし,存在性を示すだけであればそこまで難しくない.
\end{note*}

\begin{proof}
    ここでは存在性のみを示す.鍵となるのは $p^k$ もNC多項式となることである.つまり $(p^k)^{(n)}=(p^{(n)})^k$ が成り立つ.したがって, $p^k$ を展開すると $E[\mathrm{tr}(p^k)^{(n)}]$ は $n\to\infty$ で収束することがいえる.実際 $E[\mathrm{tr}(p^k)^{(n)}]$ は $E[\mathrm{tr}X_{i_1}^{(n)}\cdots X_{i_k}^{(n)}]$ の線形結合であり,各 $E[\mathrm{tr}X_{i_1}^{(n)}\cdots X_{i_k}^{(n)}]$ は $n\to\infty$ で収束する.ゆえに
    \begin{equation}
        E[\mathrm{tr}(p^{(n)})^k]=\int x^k dE\mu_{p^{(n)}}
    \end{equation}
    は任意の $k$ で収束することがいえる.ゆえに,任意の多項式について有限の極限値が存在する.
\end{proof}
\section{Preliminaries}

\begin{dfn}[a.s. bounded]
    実数値確率変数$X$がa.s. boundedとは,ある定数$M>0$が存在して,$P(|X|\leq M)=1$であることをいう.
\end{dfn}

\begin{dfn}[sub-gaussian]
    実数値確率変数$X$がsub-gaussianであるとは,ある定数$C>0,c>0$が存在して,任意の$\lambda>0$に対して以下が成り立つことをいう:
    \begin{equation}
        P(|X|\geq\lambda)\leq C\exp(-c\lambda^2)
    \end{equation}
\end{dfn}

\begin{example}
    $X\sim\mathcal{N}(0,1)$のとき,ある定数$C>0,c>0$が存在して以下のようにできる:
    \begin{equation}
        P(|X|\geq\lambda)=2\int_\lambda^\infty \frac{1}{\sqrt{2\pi}}e^{-x^2/2}dx\leq Ce^{-c\lambda^2}
    \end{equation}
\end{example}

\begin{dfn}[有限な$k$次モーメント]
    確率変数$X$が有限の$k$次モーメントを持つとは,$E[|X|^k]<+\infty$であることをいう.
\end{dfn}

\begin{note*}
    a.s. bounded $\rightarrow$ sub-gaussian $\rightarrow$ $k$次のモーメントが有限
\end{note*}

\begin{prop}
    実数値確率変数$X$がsub-gaussianならば,$\lambda\in\mathbb{R}$に対して$E[e^{\lambda X}]<+\infty$である.\\(言い換えれば,$X$のラプラス変換は$\mathbb{R}$上で定義される.)
\end{prop}

\begin{prop}
    実数値確率変数$X$に対して,以下は同値である:
    \begin{enumerate}
        \item $X$はsub-gaussianである.
        \item ある$C>0,c>0$が存在して,任意の$t\geq 0$に対して$E[e^{tX}]\leq C\exp(ct^2)$.
        \item ある$C>0$が存在して,任意の$k\geq 1$に対して$E[|X|^k]\leq(Ck)^{k/2}$.
    \end{enumerate}
\end{prop}

\begin{prop}[Markovの不等式]
    実数値確率変数$X$に対して
    \begin{equation}
        P(|X|\geq \lambda)\leq\frac{E[|X|]}{\lambda}
    \end{equation}
\end{prop}

\begin{prop}[Jensenの不等式]
    実数値確率変数$X$とconvexな関数$g$に対して
    \begin{equation}
        g(E[X])\leq E[g(X)]
    \end{equation}
\end{prop}

\begin{proof}
    $g$がconvexであるとき,$g(x)=\sup\{d(x):d(x)=ax+b \textrm{\,s.t.\,}\forall x\in\mathbb{R},d(x)\leq g(x)\}$と表現できる.このとき$E[d(X)]=d(E[X])$が成り立つ.いま,$y=E[X]$とし,$d(x)=ax+b$を$g(y)=ay+b$となるように定めると
    \begin{equation}
        \begin{split}
            g(E[X])
            & =g(y)                                     \\
            & =ay+b                                     \\
            & =d(y)                                     \\
            & =E[d(X)]                                  \\
            & \leq E[g(X)]\quad(\because d(X)\leq g(X))
        \end{split}
    \end{equation}
    となる.
\end{proof}

\section{Concentration of Measure}

\begin{thm}
    $X_1,\ldots,X_n\overset{\mathrm{i.i.d.}}{\sim}\mathcal{N}(0,1)$とし,$F:\mathbb{R}^n\to\mathbb{R}$を1-Lipschitz関数とする.このとき,ある$C>0,c>0$が存在して,任意の$\lambda>0$に対して
    \begin{equation}
        P(|F(X)-E[F(X)]|\geq\lambda)\leq C\exp(-c\lambda^2)
    \end{equation}
\end{thm}

\begin{note*}
    この現象は測度の集中とよばれ,高次元多様体ではじめてみられた.
\end{note*}

\begin{example}
    以下では$\langle\cdot,\cdot\rangle$は標準内積とする.球面$S^{n-1}\subset\mathbb{R}^n$上で$x,y$をランダム一様独立にとる.このとき,$\langle x,y\rangle\approx\frac{1}{\sqrt{n}}$であり,実際$E[|\langle x,y\rangle|^2]=\frac{1}{n}$である.

    ここで,Markovの不等式より
    \begin{equation}
        \begin{split}
            P\mleft(|\langle x,y\rangle|\geq \frac{t}{\sqrt{n}}\mright) & =P\mleft(|\langle x,y\rangle|^2\geq \frac{t^2}{n}\mright) \\
            & \leq \frac{E[|\langle x,y\rangle|^2]}{t^2/n}              \\
            & \leq\frac{1}{t^2}
        \end{split}
    \end{equation}

    これより,高次元で2つのベクトルをランダムにとれば,それらはa.s.でほぼ直交することがわかる.実際,これはsub-gaussianである.
\end{example}

\begin{note*}\,\vspace{-0.5\baselineskip}
    \begin{itemize}
        \item 例えば $F(x)=x^2$ のような Lipschitzでない関数では成立しない.
        \item $E[F(X)]$ は有界である.これは,1-Lipschitzならば $\exists C>0,|F(X)|\leq C+|X|$ に対して期待値をとり,多次元のRolleの定理を適用することでわかる.
    \end{itemize}
\end{note*}

\begin{proof}
    一般性を失わずに $E[F(X)]=0$ としてよい. そうでなければ $F(X)-E[F(X)]$ でおきかえる(1-Lipschitzを思い出す).
    $F$ と $-F$ の対称性から $P(F(X)\geq\lambda)\leq C\exp(-c\lambda^2)$ を示せば十分.さらに,$F$ が $C^1$ 級であり $|\nabla F|\leq 1$ を仮定しても一般性を失わない\footnote{
        density argumentによって,すべての1-Lipschitzな関数は $C^1$ 級で 1-Lipschitzな関数によって,sup normにおいて近似できることが示せる.具体的には以下のようになる:

        1-Lipschitzな関数 $F:\mathbb{R}^n\to\mathbb{R}$ に対して畳み込みを考える.すなわち, $\chi_\varepsilon:\mathbb{R}^n\to\mathbb{R}$ s.t. (i)$\chi_\varepsilon\geq 0$,(ii)$\chi_\varepsilon(x)=0$ if $|x|\geq\varepsilon$(smooth approximate of unit), (iii)$\int_{x\in\mathbb{R}^n}\chi_\varepsilon(x)dx=1$ となるものを考える.これを使って $F_\varepsilon\coloneqq F*\chi_\varepsilon: x\mapsto\int_{x\in\mathbb{R}^n}\chi_\varepsilon(y)F(x-y)dy$ とする.このとき$\forall\varepsilon>0,\exists F_\varepsilon:\mathbb{R}^n\to\mathbb{R}$ s.t. $F_\varepsilon$ は 1-Lipschitzかつ$C^1$ 級であり, $\forall x\in\mathbb{R},|F_\varepsilon(x)-F(x)|\leq\varepsilon$ となる.

        実際,$F$ が 1-Lipschitzであることから,$\forall x,y\in\mathbb{R}^n,|F(x-y)-F(x)|\leq|y|$ である.したがって,$|F_\varepsilon(x)-F(x)|=\mleft|\int_{x\in\mathbb{R}^n}\chi_\varepsilon(y)(F(x-y)-F(x))dy\mright|\leq\int_{x\in\mathbb{R}^n}\chi_\varepsilon(y)|F(x-y)-F(x)|dy\leq\int_{x\in\mathbb{R}^n}\chi_\varepsilon(y)|y|dy\leq\varepsilon$ となる.
    }.したがってMarkovの不等式より $P(F(X)\geq\lambda)\leq \exp(-t\lambda)E[\exp(tF(X))]$ となるので,$\forall\varepsilon\geq 0$ に対して $E[\exp(t F(X))]\leq\exp(ct^2)$ を示せば十分.これを示すために, $E[\exp(t(F(X)-F(Y)))]\leq\exp(ct^2)$ を示す(ただし $X,Y$ は独立).

    "replica" trickを用いる。すなわち,$Y=(Y_1,\ldots,Y_n)$ を $X=(X_1,\ldots,X_n)$ の独立なコピーとする.このとき,Jensenの不等式から, $E[F(Y)]=0$ に注意して
    \begin{equation}
        \exp(-tE[F(Y)])=1\leq E[\exp(-tF(Y))]
    \end{equation}
    となる. $X$ と $Y$ は独立なので, $F(X)$ と $F(Y)$ も独立で,
    \begin{equation}
        \begin{split}
            E[\exp(tF(X))]
            & =E[\exp(tF(X))]\cdot 1             \\
            & \leq E[\exp(tF(X))]E[\exp(-tF(Y))] \\
            & = E[\exp(t(F(X)))\exp((-F(Y)))]    \\
            & = E[\exp(t(F(X)-F(Y)))]
        \end{split}
    \end{equation}
    が成り立つ. $E[\exp(tF(X))]$ を評価する代わりに $E[\exp(t(F(X)-F(Y)))]$ を評価することを考える.$X_\theta=Y\cos\theta+X\sin\theta$ とおく.このとき
    \begin{equation}
        F(X)-F(Y)=\int_0^{\frac{\pi}{2}}\frac{d}{d\theta}F(X_\theta)d\theta
    \end{equation}
    である\footnote{
        ここで,微分積分学の基本定理を利用する:
        \begin{equation}
            F(X)-F(Y)=\int_0^1\frac{d}{dt}F(tX+(1-t)Y)dt
        \end{equation}
        ただし,ここでは直線の代わりに曲線を利用している:
        \begin{equation}
            F(X)-F(Y)=\int_0^{\frac{\pi}{2}}\frac{d}{d\theta}F(Y\cos\theta +X\sin\theta )d\theta
        \end{equation}
    }.$\dot{X}_\theta=\frac{d}{d\theta}X_\theta=-Y\sin\theta+X\cos\theta$\footnote{
        $X,Y\sim_{\textrm{i.i.d}}\mathcal{N}(0,I)$ とすれば $X_\theta\sim\mathcal{N}(0,\sqrt{I\cos^2\theta+I\sin^2\theta})=\mathcal{N}(0,I)$
    }なので $\frac{d}{d\theta}F(X_\theta)=\langle\nabla F(X_\theta),\dot{X}_\theta\rangle$ であり
    \begin{equation}
        \begin{split}
            \exp(t(F(X)-F(Y)))
            & =\exp\mleft(\int_0^{\frac{\pi}{2}}\langle\nabla  F(X_\theta),\dot{X}_\theta\rangle d\theta\mright)                             \\
            & =\exp\mleft(\int_0^{\frac{\pi}{2}}\langle\nabla F(X_\theta),\dot{X}_\theta\rangle \frac{\pi}{2}\frac{2}{\pi}d\theta\mright)    \\
            & \leq\frac{2}{\pi}\int_0^{\frac{\pi}{2}}\exp\mleft(\frac{\pi}{2}t\langle\nabla F(X_\theta),\dot{X}_\theta\rangle\mright)d\theta
        \end{split}
    \end{equation}
    となる.ただし,最後の不等号はJensenの不等式を用いた( $\theta$ が $[0,\frac{\pi}{2}]$ 上の一様分布のパラメータとみる).
    \begin{equation}
        \begin{split}
            E\mleft[\exp\mleft(\frac{\pi}{2}t\langle\nabla F(X_\theta),\dot{X}_\theta\rangle\mright)\mright]
            & = E\mleft[\exp\mleft(\frac{\pi}{2}t\langle\nabla F(Y),X\rangle\mright)\mright]\quad\mleft((X_\theta,\dot{X}_\theta)\overset{\mathrm{d}}{=}(X,Y)\mright)                                             \\
            & = \int_{y\in\mathbb{R}^n}\int_{x\in\mathbb{R}^n}\exp\mleft(\frac{\pi}{2}t\langle\nabla F(y),x\rangle\mright)d\mu_X(x)d\mu_Y(y)\quad\mleft(d\mu(x)=(2\pi)^{-\frac{n}{2}}e^{-\frac{x^2}{2}}dx\mright) \\
            & = \int_{y\in\mathbb{R}^n}\exp(\frac{\pi^2}{8}t^2\|\nabla F(y)\|^2)d\mu_Y(y)                                                                                                                         \\
            & \leq \exp(ct^2) \quad (\exists c>0)
        \end{split}
    \end{equation}
    したがって
    \begin{equation}
        E[\exp(tF(X))]\leq E[\exp(t(F(X)-F(Y)))]\leq\frac{2}{\pi}\int_0^{\frac{\pi}{2}}E\mleft[\exp\mleft(\frac{\pi}{2}t\langle\nabla F(X_\theta),\dot{X}_\theta\rangle\mright)\mright]d\theta\leq\exp(ct^2)
    \end{equation}
\end{proof}

\section{Eigenvalue Inequalities}

$A$ をエルミート行列とし,$\lambda_1(A)\geq\cdots\geq\lambda_n(A)$ を $A$ の固有値とする(このとき,この固有値はすべて実数である).

\begin{qes}[Horn's problem]
    $A,B$ をエルミート行列とし,$\lambda_1(A)\geq\cdots\geq\lambda_n(A)$,$\lambda_1(B)\geq\cdots\geq\lambda_n(B)$ をそれぞれの固有値とする.このとき,$A+B$ のすべての可能な固有値は何か?
\end{qes}

以下では $\lambda_i(A)=\lambda_i,\lambda_i(B)=\mu_i$ とおく.

\begin{example}
    例えば$\lambda_1,\lambda_2=-\lambda_1$ と $\mu_1,\mu_2=-\mu_1$ であり $\lambda_1\geq\mu_1\geq 0$ の場合,一般性を失うことなく $\mathrm{Tr}A=\mathrm{Tr}B=0,0\leq\lambda_2=-\lambda_1,0\leq\mu_2=-\mu_1$ とおける.このとき,Horn's problemの答えは $\forall\lambda_1,\lambda$ s.t. $\lambda\in[\lambda_1-\mu_1,\lambda_1+\mu_1]$
\end{example}

\begin{note*}[表現論との関連性]
    Clebsch-Gordan's rule: $V_\lambda\otimes V_\mu=V_{\lambda+\mu}\oplus\cdots\oplus V_{\lambda-\mu}$

    $n=2$ では $\mathrm{SU}(2)$のスピン規約表現 これは$\mathbb{R}^2$ の断片.
    一般の $n$ では,Hornは$\mathbb{R}^n$のpolytopeであると予想した.
    Kirwan Guillemin Sternberg はpolytopeであることを示した(シンプレクティック幾何)
    Klyachkoは表現論のsaturation conjecture(theorem) と等しいことを示し,KnutsonとTaoがsaturation conjectureを証明した.
\end{note*}

Horn's problem の(部分的な)説明:

$A$の固有値を$\lambda_1\geq\cdots\geq\lambda_n$ ,$B$の固有値を $\mu_1\geq\cdots\geq\mu_n$ , $A+B$ の固有値を $\nu_1\geq\cdots\geq\nu_n$ とする.このとき$\nu_1+\cdots+\nu_n=\lambda_1+\cdots+\lambda_n+\mu_1+\cdots+\mu_n$ ( $\mathrm{Tr}(A+B)=\mathrm{Tr}A+\mathrm{Tr}B$ )である.このとき $\nu_1\leq \lambda_1+\mu_1$を示す:

定数だけシフトすることで $\lambda_n,\mu_n\geq 0$ として考えてよい.もし $A,B\geq O$(正定値)ならば $A+B\geq O$ .これを仮定すると$\lambda_1=\|A\|,\mu_1=\|B\|,\nu_1=\|A+B\|$であり,$\nu_1\leq \lambda_1+\mu_1$ は作用素ノルムの三角不等式となる.

% 続き書く

\section{Stein's Method}

%  [Tao, 2.2.6]

ガウス分布 $\mathcal{N}(0,1)$ と, ガウス分布に分布として近づいていくものの特徴づけを与える(ガウス分布に近い分布はどうなるか).

標準ガウス分布の確率密度関数 $\rho(x)=\frac{1}{\sqrt{2\pi}}e^{-\frac{x^2}{2}}$ は微分方程式 $\rho'(x)+x\rho(x)=0$ をみたす.いま $f:\mathbb{R}\to\mathbb{R}$ を $C^1$ 級,$\exists C>0$ s.t. $\forall x\in\mathbb{R},|f(x)|\leq C,|f'(x)|\leq C$ とすると
\begin{equation}
    \int_{-\infty}^{\infty}\rho(x)(xf(x)-f'(x))dx=0
\end{equation}
が成り立つ.これを確率論の言葉で書き直すと,$X\sim\mathcal{N}(0,1)$ に対して
\begin{equation}
    E[Xf(X)]=E[f'(X)]
\end{equation}
が成り立つ.逆も成り立つことが知られている:

\begin{lem}
    $X$ を実数値確率変数であって $\forall f:\mathbb{R}\to\mathbb{R}$ であり $E[X f(X)]=E[f'(X)]$ をみたす(*)ならば, $f(X)\sim\mathcal{N}(0,1)$
\end{lem}

したがって,この恒等式 $E[Xf(X)]=E[f'(X)]$ を(正確に等号でなくても)ほとんどみたせば,$X$ はほとんどガウス分布に従うことがわかる.

分布$X$が何かわからない状態で,$G$に近いことを示すには?
\begin{itemize}
    \item $X$ を $X\cos\theta+G\sin\theta$ に置き換える($X$ は未知の分布,$G\sim\mathcal{N}(0,1)$, $G$は$X$と独立).後で $\theta\to 0$ をとる.
    \item モーメントを調べる(有界でない場合は適当にtruncateする).
\end{itemize}

今回のケースではモーメントが一意に分布を決定する(Hamburger moment problem).

\begin{thm}[Stein continuity theorem]
    $\{X_n\}$ を $E[|X_n|^2]<\infty$ な確率変数列とする.このとき,以下は同値である:
    \begin{enumerate}
        \item 任意の$f$ s.t. $C_1$ 級, $f,f'$ は有界に対して $E[f'(X_n)-X_nf(X_n)]\to 0$
        \item $X_n\overset{\mathrm{d}}{\longrightarrow}G$($G\sim\mathcal{N}(0,1)$)
    \end{enumerate}
\end{thm}

\begin{proof}
    概略を示す.

    \textbf{[(ii)$\to$(i)]} $\varphi:\mathbb{R}\to\mathbb{R}$ は有界連続とする.$E\varphi(X_n)\to E\varphi(G)$ を示したい.$\varphi$が(*)をみたせばこれが成り立つ.$E[\varphi'(X_n)-X_n\varphi(X_n)]\to E[\varphi'(G)-G\varphi(G)]=0$ を示したい.$\varphi$ が(*)をみたすかは不明なので,近似を行う.
    \begin{enumerate}
        \item $\varphi$ を $\varphi_\varepsilon$ によって近似($|\varphi-\varphi_\varepsilon|<\varepsilon,|\varphi_\varepsilon'|<10/\varepsilon$).
        \item $x\mapsto x\varphi(x)$ を任意の区間でコンパクトな有界な関数によって近似する.
    \end{enumerate}
    こういった理由により $\limsup_{n\to\infty}|E[f'(X_n)-X_nf(X_n)]|<100\varepsilon$.

    \textbf{[(i)$\to$(ii)]} $\varphi$ を有界とする. $E\varphi(X_n)\to E\varphi(G)$ を示せば十分.一般性を失わずに $E\varphi(G)=0,|\varphi(x)|\leq 1(\forall x)$ としてよい.ここで,$f:\mathbb{R}\to\mathbb{R}$ s.t. $\forall x,\varphi(x)=f'(x)-xf(x)$ を見つけたい.
    \begin{enumerate}
        \item $f(x)=e^{\frac{x^2}{2}}\int_{-\infty}^x e^{-\frac{y^2}{2}}\varphi(y)dy$ をとる.
        \item $\varphi(X_n)=f'(X_n)-X_nf(X_n)$ となっていることを確認する.
    \end{enumerate}
    このとき $E[\varphi(X_n)]=E[f'(X_n)-X_nf(X_n)]\to E[G]=0$ となるのでよい.

\end{proof}

\section{The operator norm of random matrices}

Wiegner's theorem は 暗に次のことを意味している:$2<a<b$ とし,確率変数 $\#\{i;\lambda_i^{(n)}\in[a,b]\}=n\mu_{X^{(n)}}([a,b])=\rho_n([a,b])\in\{0,\ldots,n\}$を考える.$E[\rho_n([a,b])]=n\int_a^b ds=o(n)$

\begin{qes}
    固有値 $\lambda_i^{(n)}$ s.t. $|\lambda_i^{(n)}|>2+\varepsilon$ となるものは存在するのか?言い換えれば,確率変数 $\lambda_1^{(n)},\lambda_n^{(n)}$ は極限をもつのか?もしそうなら $+2,-2$ となるのか?
\end{qes}

類似の問題

\begin{qes}
    $\max\{|\lambda_1^{(n)}|,|\lambda_n^{(n)}|\}=\|X^{(n)}\|=\sup_{|\lambda|\neq 0}\|X\lambda\|_{L^2}/\|\lambda\|_{L^2}$ のとき,$\lim_{n\to\infty}\|X^{(n)}\|$ はどうなるか?
\end{qes}

次のことは容易に示せる

\begin{lem}
    任意の $\varepsilon>0$ に対して $P(\liminf_{n\to\infty}\|X^{(n)}\|\geq 2)\geq 1-\varepsilon$
\end{lem}

\begin{proof}
    (概略)もし $\liminf_{n\to\infty}\|X^{(n)}\|<2-\varepsilon$ ならば,全固有値は $[-2+\varepsilon,2-\varepsilon]$ に入っている.$\mathrm{tr}((X^{(n)})^{2k})\to\mathrm{Cat}(k)\leq (2-\varepsilon)^{2k}$ だが, $\mathrm{Cat}(k)\sim c\cdot 4^k\cdot k^{-\frac{3}{2}}$ なので,十分大きな $k$ で矛盾する.
\end{proof}

これより,$P(\limsup_{n\to\infty}\|X^{(n)}\|\leq 2)=1$ がいえれば,$\lim_{n\to\infty}\lambda_1^{(n)},\lim_{n\to\infty}\lambda_n^{(n)}$の存在と,それらが $+2,-2$ であることを暗に示す.これを示すには,2つのアプローチがある:

\begin{enumerate}
    \item "soft" net argument:$P(\|X^{(n)}\|\geq 10)\to 0 (n\to\infty)$(boundが2だとダメだが,5倍すると示せる).なお $\mathrm{supp} L(\|X^{(n)}\|)=[0,\infty]$
    \item strong moment argument:$\forall\varepsilon>0,P(\limsup_{n\to\infty}\|X^{(n)}\|>2+\varepsilon)=0$
\end{enumerate}

\subsection{"soft" net argument}

簡単のため $\sqrt{n}X^{(n)}$ について調べる.$X_{ij}^{(n)}$ は $n$ に依存した分散をもつ.$S=\{x\in\mathbb{C}^n:|\lambda|=1\}$ とする.このとき,十分に大きな定数 $A$ と $n$ によらない定数 $c,C$ があって
\begin{equation}
    \forall x\in S,P(|Mx|\geq \sqrt{n}A)\leq C\exp(-ncA)\tag{*}
\end{equation}
となる(Wick's theoremなどを利用して愚直に分散を計算すると示せる).

\begin{note*}
    実際,この結果は$\sqrt{n}X_{ij}^{(n)}$がiidのサブガウシアンである場合と同じである.
\end{note*}

\begin{lem}
    $\Sigma$ を $S$ のmaximal $1/2$-net とする.このとき$\mathbb{C}$ 値 $n\times n$ ランダmム行列 $X^{(n)}$ について%このとき $\Sigma\subset S$ は$\varepsilon$-net である,すなわち $x,y\in\Sigma,x\neq y\Rightarrow |x-y|\geq \varepsilon$
    \begin{equation}
        \forall\lambda>0,P(\|X_1^{(n)}\|>\lambda)\leq P\mleft(\bigvee_{y\in\Sigma}|X^{(n)}y|>\frac{\lambda}{2}\mright)
    \end{equation}
    が成り立つ.
\end{lem}

\begin{note*}
    左辺の条件は,$\|X^{(n)}\|=\sup_{x\in S}\|X^{(n)}x\|>\lambda\Leftrightarrow \forall x\in S,|X^{(n)}x|>\lambda$.このままでは無限個の $x$ を考えなければならないが,右辺を評価することで有限個について考えれば済むことになる.こうして分解した右辺を,上で述べた(*)式を利用して抑える.
\end{note*}
\begin{proof}
    $x\in S$ を $\|M\|=|Mx|$ ととる.また $y\in\Sigma$ を $|x-y|<1/2$ ととる.このとき $|Mx-My|\leq\|M\|/2$.三角不等式から $|My|>\|M\|/2$ が従う.よって, $\|M\|>\lambda$ ならば $|My|>\lambda/2$.
\end{proof}

$P\mleft(\bigvee_{y\in\Sigma}|X^{(n)}y|>\lambda/2\mright)\leq\#\Sigma\cdot\max_{y\in\Sigma}P(|X^{(n)}y|>\lambda/2)$ が成り立つことは直ちにわかる.ここで $\#\Sigma$ を評価する.

\begin{lem}
    $\Sigma$ を $S$ の $\varepsilon$-netとする.このとき,$\exists c>0$ s.t. $\#\Sigma <(c/\varepsilon)^n$
\end{lem}

\begin{proof}
    $\#\Sigma\cdot\mathrm{Vol} B(0,\varepsilon/2)\leq\mathrm{Vol}(B(0,3/\varepsilon))$.ただし $\mathrm{Vol}(B(0,x))=Cx^n,\exists C>0$.
\end{proof}

以上より,(*)も合わせて
\begin{equation}
    P(\|X^{(n)}\|>\lambda)\leq\tilde{C}^n C\exp\mleft(-nc\cdot\frac{\lambda}{2}\mright)
\end{equation}
$\lambda$ を十分大きくとれば,右辺は指数的に0に近づく.

\begin{note*}
    上記のsoft "ε-net" argument では $\exists C> 0$ があって, $P(\|X^{(n)}\|>C)\overset{n\to\infty}{\longrightarrow} 0$ となる.この方法では,$C$は有限だが $C=10$ となってしまい最適化できない.
\end{note*}

\subsection{strong moment argument}

\begin{thm}
    任意の $\varepsilon>0$ に対して $P(\|X^{(n)}\|> 2+\varepsilon)\overset{n\to\infty}{\longrightarrow}0$ ならば, $P(\limsup_{n\to\infty}\|X^{(n)}\|\leq 2+\varepsilon)=1$ となる.
\end{thm}

\begin{note*}
    この結果は,$P(\liminf_{n\to\infty}\|X^{(n)}\|\leq 2-\varepsilon)=1$ とあわせると $P(\lim_{n\to\infty}\|X^{(n)}\|=2)=1$ であることがわかる.
\end{note*}

$k\in\mathbb{N}$ をとる. $\|(X^{(n)})^{2k}\|=\|X^{(n)}\|^{2k}\leq\mathrm{Tr}((X^{(n)})^{2k})$.期待値をとると $E[\|X^{(n)}\|^{2k}]\leq E[\mathrm{Tr}((X^{(n)})^{2k})]$ となる.また,Markovの不等式より
$P(\|X^{(n)}\|^{2k}>2+\varepsilon)\leq E[\|X^{(n)}\|^{2k}]/(2+\varepsilon)^{2k}$.

ここで $\mathrm{Inv}(2k)=\{\sigma\in S_{2k};\sigma^2=\mathrm{id},[\forall x\in\{1,\ldots,2k\},\sigma(x)\neq x]\}$ (後ろの条件は大体pair partition)として, $f(k,i)=\#\{\sigma\in\mathrm{Inv}(2k)\,\mathrm{s.t.}\,|\gamma|+2i=|\gamma\sigma|+|\sigma|,\gamma=(1\,2\,\ldots\,2k)\in S_{2k}\}$とする.

\begin{lem}
    $f(k,i+1)\leq k^2f(k,i)$
\end{lem}

\begin{proof}
    % 手書きノートの$z=(1\,2\,\ldots\,2k)\in S_{2k}$ だが、latexでは\gammaにしてる
    $\sigma\in S_{2k}$ をとる.このgenks $g(\sigma)$ によって $|\gamma|+g(\sigma)=|\gamma\sigma|+|\sigma|$.すべてのケーリーグラフ$\{\sigma':|\sigma\sigma'|=1\}$ 内の $\sigma$ のneighborに注目する.もし $g(\sigma)\neq 0$ なら,少なくとも1つのneighborは $g(\sigma')=g(\sigma)$ をみたす.
\end{proof}

これを使うと,

\begin{equation}
    \begin{split}
        P(\|X^{(n)}\|^{2k}>2+\varepsilon)
        &\leq (2+\varepsilon)^{-2k}\mleft\{n\mathrm{Cat}(k)+n^{-1}k^2\mathrm{Cat}(k)+n^{-3}k^4\mathrm{Cat}(k)+\cdots\mright\}\\
        &\leq (2+\varepsilon)^{-2k}\mathrm{Cat}(k)\cdot n\mleft\{1+\mleft(\frac{k}{n}\mright)^2+\mleft(\frac{k}{n}\mright)^4+\cdots\mright\}\\
        &\leq (2+\varepsilon)^{-2k}\mathrm{Cat}(k)\cdot n\cdot 2
    \end{split}
\end{equation}

スターリングの公式より $\mathrm{Cat}(k)\sim 4^k k^{-\frac{3}{2}}(1+o(1))$(あるいは $\mathrm{Cat}(k)\leq 4^k/(k+1)$ を用いて $P(\|X^{(n)}\|^{2k}>2+\varepsilon)\leq (2+\varepsilon)^{-2k}\frac{4^k}{k+1}\cdot n\cdot 2\leq (\frac{2+\varepsilon}{\varepsilon})^{-2k}\frac{2n}{k}$).
もし $k\gg\log n$ なら右辺は $0$ に収束する.

\begin{note*}
    $n\times n$ 複素正方行列 $X$ に対してそのscatter $p$-normは $\|X\|_p=\sqrt[p]{\mathrm{Tr}((XX^*)^{p/2})}$ で定義される.ここで $\|X\|\leq\|X\|_p$ を使う.(ここで$n^{-p}$だけ違うが,$\sqrt[p]{\mathrm{tr}((XX^*)^{p/2})}\leq\|X\|$もまた成り立つ.したがって $n^{-1/p}\|X\|_p\leq\|X\|\leq\|X\|_p$ が成り立つ.) $n^{-1/p}\to 1$ は $p\gg\log n$ と同値.
\end{note*}

\section{Stieltjes transform}

確率論による一般原理:変換を通した分布の分析.例えば...
\begin{itemize}
    \item フーリエ変換: $F_X(t)=E[e^{itX}]$
    \item ラプラス変換: $L_X(t)=E[e^{tX}]$
    \item メリン変換: $M_X(t)=\int_{-\infty}^{\infty}e^{itx}dF_X(x)$
    \item ...
\end{itemize}

Stieltjes transformはWiegner's semicircular theoremの新たな証明を与えた.

\begin{dfn}
    実数値確率測度 $\mu$ に対して,そのStieltjes transformは $z\in\mathbb{C}\backslash\mathbb{R}$ に対して
    \begin{equation}
        S_\mu(z)=\int\frac{1}{x-z}d\mu(x)
    \end{equation}
    で定義される.
\end{dfn}

$|1/(x-z)|\leq1/|\mathrm{Im}z|$ ならば, $S_\mu$ はwell-definedであり, $|S_\mu(z)\leq 1/|\mathrm{Im}z|$ をみたす.

\begin{note*}
    歴史的な注意.Cauchy変換
    \begin{equation}
        G_\mu(z)=\int_\mathbb{R}\frac{1}{z-x}d\mu(x)
    \end{equation}
    とは $G_\mu(z)=-S_\mu(z)$ の関係がある.Stieltjesは確率論を意識してこの関数を定義したが,Cauchyはそのことを考えていたかは不明.
\end{note*}

$X$ を分布 $\mu$ に従う実数値確率変数とする.また $\mathrm{supp}\mu\subset [-K,K]$ であり $z$ は $|z|>K$ にとることにする.いま $G_\mu(z)$ を
\begin{equation}
    G_\mu(z)=\sum_{k=0}^\infty\frac{E[X^k]}{z^{k+1}}
\end{equation}
とかく.もし $\mu$ が有界なサポートをもてば,すべてのモーメントは有限である.モーメントは $\mu$ を一意に決定づける.これによって $\int p(x)d\mu(x)$ が計算可能となる.したがってstone-weierstrassの定理によって有界連続な関数 $\rho(x)$ に対して $\int\rho(x)d\mu(x)$ が計算可能となる.ゆえに Risezの表現定理によって $\mu$ は一位に特徴付けられる.

よって, $S_\mu(z)$ は $\mu$ を特徴づける.もし $\mu$ が有界サポートをもつなら $\mu\mapsto S_\mu$ の単射がつくれる.$X$ のモーメントは $z\to\infty$ での $G_\mu=-S_\mu$ のテイラー展開の係数によって決まる.

\begin{note*}
    もし $\mu$ が単純点測度 $\mu=\sum_{i=1}^\infty \alpha_i\delta_{\lambda_i}$ なら, $\mu$ は容易に $S_\mu$ から回復できる.
\end{note*}

\begin{example}
    $\mu=\delta_0/3+\delta_1/3+\delta_2/3$ なら,
    \begin{equation}
        S_\mu(z)=\frac{1}{3}\mleft(\frac{1}{z}+\frac{1}{1-z}+\frac{1}{2-z}\mright)
    \end{equation}
\end{example}

一般の一意な特徴づけ
まず,次を考える
\begin{equation}
    \mathrm{Im} S_\mu(a+ib)=\pi\mu * P_b(a)
\end{equation}
ここで,$P_b(a)$ は $P_b(a)=\frac{1}{\pi}\frac{b}{(a-x)^2+b^2}=\frac{1}{b}P_1\mleft(\frac{a}{b}\mright)$ である.

\begin{example}
    $d\mu=\rho(x)dx$ とすると $\rho(x)=\lim_{\varepsilon\to+0}\mathrm{Im}\frac{1}{\pi}S_\mu(x+i\varepsilon)$
\end{example}

$S_\mu$ を半円を示すのにどうやって使うか?

$X_n$を$\mathrm{GUE}(n)$ とする.$z$ を $\mathrm{Im}z>0$ にとる.このとき,
\begin{equation}
    s_n(z)\coloneqq E\mleft[\mathrm{tr}((X_n-zI_n)^{-1})\mright]=S_{\mu_n}(z)
\end{equation}

ここで $\mu_n$ は正規化した期待固有値計数測度

これらを利用して $s_n(z)\to s_{\mu_{\mathrm{sc}}}(z)$ を示す.

メインアイデアは,$X_n$ に対する $s_n(z)$ と,その左上 $(n-1)\times(n-1)$小行列 $\tilde{X}_n$ に対する $s_n(z)$ を比較することである.

Exercise 2.4.11

これを $X=X_n-zI$ に適用して期待値をとると

\begin{equation}
    s_n(z)=-E\mleft[\frac{1}{z-o(1/\sqrt{n})-B_n^*(\tilde{X}_n-zI_{n-1})B_n}\mright]
\end{equation}

ここで $E[\mathrm{tr}((X_n-zI_n)^{-1})]=E[((X_n-zI_n)^{-1})_{11}]$ である.

2. 実際に以下を示せる:
\begin{equation}
    |B_n^*(\tilde{X}_n-zI_{n-1})B_n-s_n(z)|=o\mleft(\frac{1}{n}\mright)
\end{equation}

algebric argumentとして $\tilde{\lambda}^{(n)}_1\geq\cdots\geq\tilde{\lambda}^{(n)}_{n-1}$ は $\tilde{X}_n$ の固有値とすると, $\lambda^{(n)}_i\geq\tilde{\lambda}^{(n)}_i\geq\lambda^{(n)}_{i+1}$が成り立つ.

最後に, $B_n$ と $\tilde{X}_n-zI_{n-1}$ が独立であることを使うと $B$ が消える.以上により結局
\begin{equation}
    s_n(z)=-\frac{1}{z+s_n(z)}+o(1)
\end{equation}
がわかる.$s_n$ は $s_{\mu_{\mathrm{sc}}}(z)=-\frac{1}{z+s_{\mu_{\mathrm{sc}}}(z)}$ の解に収束しなければならない.これで求まる $s(z)$ は,半円分布のStieltjes transformになっている:
\begin{equation}
    s_{\mu_{\mathrm{sc}}}(z)=\frac{-z+\sqrt{z^2-4}}{2}
\end{equation}

\end{document}
