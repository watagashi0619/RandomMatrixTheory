\documentclass{ltjsarticle}
\usepackage[
    pdfencoding=auto,
    psdextra,
    pdfusetitle,
    hidelinks,
    pdfversion=1.7,
    pdfa
]{hyperref}%ハイパーリンク
\usepackage{hyperxmp}
\hypersetup{
    pdfsubject={解析学特論Ⅱの講義に基づいてまとめたもの},
    pdfkeywords={ランダム行列},
    pdfapart=3, % PDF/A のバージョン。最新は3だが、実際は2で問題ないかも
    pdfaconformance=u, % PDF/A 内のLevel区分。b(basic), a(accesible), u(unicode)が存在
    colorlinks=true,
    linkcolor=red,
}
% カラープロファイルを埋め込んで、出力インテントの設定をしているっぽい(https://tex.stackexchange.com/a/534201)
\immediate\pdfobj stream attr{/N 3} file{sRGB.icc}
\pdfextension catalog{
    /OutputIntents [
        <<
            /Type /OutputIntent
            /S /GTS_PDFA1
            /DestOutputProfile \the\pdflastobj\space 0 R
            /OutputConditionIdentifier (sRGB)
            /Info (sRGB)
        >>
    ]
}
%\pdfvariable omitcidset=1 %デフォルト以外のフォントを読み込む場合、この一行が必要になる場合がある
%\usepackage[1.7]{bxpdfver}
\usepackage{amsthm}%定理環境
\usepackage[hiragino-pron]{luatexja-preset}
\usepackage{newtxtext,newtxmath}%times font setting
\usepackage{amsmath,amsfonts}%数式font setting
\usepackage{mathrsfs}%花文字
\usepackage{pxrubrica}%ルビ
%\usepackage{lmodern}%font setting
\usepackage{mathtools}%数式の相互参照部分にのみ式番号をふる
\mathtoolsset{showonlyrefs,showmanualtags}
\usepackage{empheq}%方程式
\usepackage{physics}%物理系
\usepackage{bm}%vector
\usepackage{mleftright}%カッコ
\usepackage{framed}%フレーム
\usepackage{color}%文字に色付け
\usepackage{ulem}%取り消し線
\usepackage{bbm}%指示関数とかはこっちで出す
%\usepackage[right]{showlabels}%数式番号を左に表示
\usepackage{booktabs}%表の上線\toprule中線\midrule下線\bottomrule
%\usepackage{url}%参考文献とかでurlをかくとき
%\usepackage{comment}%長文コメント
%\usepackage{lscape}%ページを横向きにする
\usepackage{ascmac}
\usepackage{tcolorbox}
\tcbuselibrary{theorems,skins}%数式を「デコ」る
\usepackage{tikz}
\usetikzlibrary{patterns}
\usetikzlibrary{cd}%矢印の図
\usetikzlibrary{decorations,decorations.pathreplacing}%tikzでデコる
% patternsの斜線設定
\pgfdeclarepatternformonly{south west lines}{\pgfqpoint{-0pt}{-0pt}}{\pgfqpoint{3pt}{3pt}}{\pgfqpoint{3pt}{3pt}}{
    \pgfsetlinewidth{0.4pt}
    \pgfpathmoveto{\pgfqpoint{0pt}{0pt}}
    \pgfpathlineto{\pgfqpoint{3pt}{3pt}}
    \pgfpathmoveto{\pgfqpoint{2.8pt}{-.2pt}}
    \pgfpathlineto{\pgfqpoint{3.2pt}{.2pt}}
    \pgfpathmoveto{\pgfqpoint{-.2pt}{2.8pt}}
    \pgfpathlineto{\pgfqpoint{.2pt}{3.2pt}}
    \pgfusepath{stroke}
}
%subsubsubsection定義 paragraphも用いる場合は注意
\makeatletter
    \newcommand{\subsubsubsection}{\@startsection{paragraph}{4}{\z@}%
        {1.0\Cvs \@plus.5\Cdp \@minus.2\Cdp}%
        {.1\Cvs \@plus.3\Cdp}%
        {\reset@font\sffamily\normalsize}
    }
\makeatother

\setcounter{secnumdepth}{3}%セクションの深さレベル

%physicsでrotもcurlで使う
\newcommand{\rot}{\curl}

%amsthm style1,2
%style1 連番
\newtheoremstyle{mystyle1}% Name
    {}% Space above
    {}% Space below
    {\normalfont}% Body font
    {}% Indent amount
    {\bfseries\sffamily}% Theorem head font
    {\hspace{0.5em}}% Punctuation after theorem head
    { }% Space after theorem head, ‘ ‘, or \newline
    {\thmname{#1}\thmnumber{#2}\thmnote{(#3)\\}}% Theorem head spec (can be left empty, meaning `normal')
\theoremstyle{mystyle1}
\newtheorem{dfn}{定義}[section]
\newtheorem{thm}[dfn]{定理}
\newtheorem{axi}[dfn]{公理}
\newtheorem{cor}[dfn]{系}
\newtheorem{prop}[dfn]{命題}
\newtheorem{lem}[dfn]{補題}
\newtheorem{exs}[dfn]{例題}

%style2 連番なし
\newtheoremstyle{mystyle2}% Name
    {}% Space above
    {}% Space below
    {\normalfont}% Body font
    {}% Indent amount
    {\bfseries\sffamily}% Theorem head font
    {\hspace{0.5em}}% Punctuation after theorem head
    { }% Space after theorem head, ‘ ‘, or \newline
    {\thmname{#1}\thmnote{(#3)\\}}% Theorem head spec (can be left empty, meaning `normal')
\theoremstyle{mystyle2}
\newtheorem{dfn*}{定義}
\newtheorem{thm*}{定理}
\newtheorem{prop*}{命題}
\newtheorem{ex}{例題}
\newtheorem{example}{例}
\newtheorem{qes}{問題}
\newtheorem{note*}{注意}
\newtheorem{ans}{解答}
\newtheorem{note}{注意}
\newtheorem{lem*}{補題}

\newtheoremstyle{mystyle3}% Name
    {}% Space above
    {}% Space below
    {\normalfont}% Body font
    {}% Indent amount
    {\bfseries\sffamily}% Theorem head font
    {\hspace{0.5em}}% Punctuation after theorem head
    { }% Space after theorem head, ‘ ‘, or \newline
    {\thmname{#1}\thmnumber{#2}}% Theorem head spec (can be left empty, meaning `normal')
\theoremstyle{mystyle3}
\newtheorem{qes*}{問題}

%proofの後ろに黒四角をつける
\makeatletter
\renewenvironment{proof}[1][\proofname]{\par
  \pushQED{\qed}%
  \normalfont
  \topsep6\p@\@plus6\p@ \trivlist
  \item[\hskip\labelsep{\bfseries\sffamily #1}]\ignorespaces
}{%
  \popQED\endtrivlist\@endpefalse
}
\renewcommand\proofname{証明}
\renewcommand{\qedsymbol}{\ensuremath{\blacksquare}}
\makeatother

%ansの後ろに黒四角をつける
\makeatletter
\renewenvironment{ans}[1][解答]{\par
  \pushQED{\qed}%
  \normalfont
  \topsep6\p@\@plus6\p@ \trivlist
  \item[\hskip\labelsep{\bfseries\sffamily #1}]\ignorespaces
}{%
  \popQED\endtrivlist\@endpefalse
}
\renewcommand{\labelenumi}{\ensuremath{\blacksquare}}
\makeatother

%方程式番号にセクション番号を入れる
\makeatletter
\@addtoreset{equation}{section}
\def\theequation{\thesection.\arabic{equation}}% renewcommand でもOK
\makeatother

% 図番号にセクション番号入れる
\makeatletter
\renewcommand{\thefigure}{
\arabic{figure}}
\@addtoreset{figure}{section}
\makeatother

%\newcommand\range{\operatorname{range}}
%\newcommand\sgn{\operatorname{sgn}}

%\renewcommand{\thepart}{\arabic{part}}
%\renewcommand{\thenote}{}
%\renewcommand{\thelem}{}

%\makeatletter
%\@addtoreset{section}{part}
%\makeatother

%\makeatletter
%\def\blfootnote{\xdef\@thefnmark{}\@footnotetext}
%\makeatother

%\setcounter{tocdepth}{1}%目次をsectionまでの表示にする

\renewcommand{\labelenumi}{(\arabic{enumi})}%enumerateのデフォルトを(1)などにする

%その他文字設定
\newcommand{\defLeftrightarrow}{\overset{\text{def}}{\iff}}

\title{解析学特論Ⅱ 講義ノート}
\author{}
\date{}
\begin{document}
\maketitle

\section{Gaussian Unitary Ensemble}

$V$は$\mathbb{R}$上の有限次元ベクトル空間であり,各要素は$r\in\mathbb{R}$として$\mathcal{N}(0,r)$に従う実数確率変数であるとする.ただし,$r=0$の場合も確率変数$0$とみなして許容する.

\begin{example}
    $g_1,\ldots,g_d\overset{\textrm{i.i.d.}}{\sim}\mathcal{N}(0,1)$としたとき,$V=\{\sum_{i=1}^d \lambda_ig_i\mid (\lambda_1,\ldots,\lambda_d)\in\mathbb{R}^d\}$はgaussian space.実際,すべてのgaussian spaceはこのように表現できる.
\end{example}

\subsection{Wick's Theorem}

$g_1,\ldots,g_n\in V$を考え,$E[g_1\cdots g_n]$の値がどうなるかを考えたい.$E[g_1\cdots g_n]$が有限であることはHölderの不等式から導ける.また,$n$が奇数の場合にこの値が0となることは容易にわかる.

\begin{dfn}[pair partition $\mathcal{P}_2(n)$]\
    \vspace{-\baselineskip}
    \begin{enumerate}
        \item $n\in\mathcal{N}$に対して$[n]=\{1,\ldots,n\}$と表記する.
        \item $[n]$のpairing $\pi$とは,$[n]$を被りのないように2つずつ選んだ数のペアに分割することである.すなわち,$\pi=\{V_1,\ldots,V_k\}$は$i,j=1,\ldots,k$(ただし$i\neq j$)に対して
              \begin{itemize}
                  \item $V_i\subset [n]$
                  \item $|V_i|=2$
                  \item $V_i\cap V_j=\emptyset$
                  \item $\bigcup_{i=1}^k V_i=[n]$
              \end{itemize}
              をみたす.なお,必然的に$k=n/2$となる.
        \item $[n]$のpair partition $\mathcal{P}_2(n)$は
              \begin{equation}
                  \mathcal{P}_2(n)=\{\pi\mid\pi\textrm{\,is\,pairing\,of\,}[n]\}
              \end{equation}
              で定義される.
    \end{enumerate}
\end{dfn}

\begin{prop}
    \begin{equation}
        \#\mathcal{P}_2(n)=\begin{cases}
            0       & (nは奇数) \\
            (n-1)!! & (nは偶数)
        \end{cases}
    \end{equation}
\end{prop}

\begin{proof}
    $\mathcal{P}_2(n)$のうちで,まず$1$とのペアを考えると,その選び方は$n-1$通りある.このペアを取り除くと,pairingを作る数は$n-2$個残っている.したがって$\#\mathcal{P}_2(n)=(n-1)\cdot\#\mathcal{P}_2(n-2)$が成り立つ.また,$\#\mathcal{P}_2(1)=0$,$\#\mathcal{P}_2(2)=1$であることと合わせれば,証明は終了する.
\end{proof}

\begin{example}
    $\mathcal{P}_2(2)=\{\{\{1,2\}\}\}$,$\mathcal{P}_2(4)=\{\{\{1,2\},\{3,4\}\},\{\{1,3\},\{2,4\}\},\{\{1,4\},\{2,3\}\}\}$.
\end{example}

\begin{dfn}
    $E_\pi [X_1,\ldots,X_n]=\prod_{(i,j)\in\pi}E[X_iX_j]$
\end{dfn}

\begin{example}
    \begin{equation}
        \begin{split}
            E[X_1X_2X_3X_4]&=E_{\{\{1,2\},\{3,4\}\}}[X_1,X_2,X_3,X_4]+E_{\{\{1,3\},\{2,4\}\}}[X_1,X_2,X_3,X_4]+E_{\{\{1,4\},\{2,3\}\}}[X_1,X_2,X_3,X_4] \\
            &=E[X_1X_2]E[X_3X_4]+E[X_1X_3]E[X_2X_4]+E[X_1X_4]E[X_2X_3]
        \end{split}
    \end{equation}
\end{example}

\begin{lem}
    $V$を$\mathbb{R}$上のベクトル空間として,$n$重線形対称関数$\varphi:V^n\to\mathbb{R}$を考える.このとき,$\varphi(x,\cdots,x)=\phi(x)$とおくと
    \begin{equation}
        \varphi(x_1,\ldots,x_n)=\frac{1}{n!}\sum_{k=1}^n\sum_{1\leq j_1<\cdots<j_k\leq n}(-1)^{n-k}\phi(x_{j_1}+\cdots+x_{j_k})
    \end{equation}
    が成り立つ.
    %\url{https://math.stackexchange.com/questions/481167/polarization-formula}
\end{lem}

\begin{note*}
    上の補題において対称とは,$\sigma\in S_n$に対して$\varphi(x_1,\ldots,x_n)=\varphi(x_{\sigma(1)},\ldots,x_{\sigma(n)})$が成り立つことである.
\end{note*}

\begin{note*}
    この等式は,偏極恒等式の一般化である.$n=2$の場合は偏極恒等式
    \begin{equation}
        \varphi(x,y)=\frac{1}{2}(\phi(x+y)-\phi(x)-\phi(y))=\frac{1}{4}(\phi(x+y)-\phi(x-y))
    \end{equation}
    となる.
\end{note*}

\begin{proof}
    \begin{equation}
        \begin{split}
            \sum_{k=1}^n\sum_{1\leq j_1<\cdots<j_k\leq n}(-1)^k\phi(x_{j_1}+\cdots+x_{j_k})
            &=\sum_{A\subset[n]}(-1)^{|A|}\phi\mleft(\sum_{a\in A}x_a\mright)\\
            &=\sum_{A\subset [n]}(-1)^{|A|}\sum_{f:[n]\to A}\varphi(x_{f(1)},\ldots,x_{f(n)}) \\
            &=\sum_{f:[n]\to[n]}\varphi(x_{f(1)},\ldots,x_{f(n)})\sum_{\mathrm{Im}f\subset A\subset [n]}(-1)^{|A|} \\
            &=\sum_{f\in S_n}\varphi(x_{f(1)},\ldots,x_{f(n)})(-1)^n \\
            &=(-1)^nn!\varphi(x_1,\ldots,x_n)
        \end{split}
    \end{equation}
    となることから従う.ただし,$\mathrm{Im}f\neq [n]$のときは
    \begin{equation}
        \begin{split}
            \sum_{\mathrm{Im}f\subset A\subset [n]}(-1)^{|A|}
            &=\sum_{j=0}^{n-|\mathrm{Im}f|}(-1)^j{}_{n-|\mathrm{Im}f|}C_j \\
            &=(1+(-1))^{n-|\mathrm{Im}f|} \\
            &=0
        \end{split}
    \end{equation}
    となることを利用した.
\end{proof}

\begin{lem}
    $n$重線形対称関数$\phi:V^n\to\mathbb{R}$に対して,任意の$x_1,\ldots,x_n\in V$に対して$\varphi(x_1,\ldots,x_n)=0$が成り立つことと,任意の$x\in V$に対して$\varphi(x,\ldots,x)=\phi(x)=0$が成り立つことは同値.
\end{lem}

\begin{proof}
    十分性は容易.必要性は先の補題において右辺に現れる各$\phi(x_{j_1}+\cdots+x_{j_k})=0$となるときを考えることになるため,$\varphi(x_1,\ldots,x_n)=0$が成り立つ.
\end{proof}

\begin{thm}[Wick]
    $g_1,\ldots,g_n\in V$に対して,以下が成り立つ:
    \begin{equation}
        E[g_1\cdots g_n]=\sum_{\pi\in\mathcal{P}_2(n)}E_\pi[g_1,\ldots,g_n]
    \end{equation}
\end{thm}

\begin{proof}
    まずは$g_1=\cdots=g_n=g\in V$のときに成り立つことを示す.いま,$V$の仮定から$E[g]=0$である.さらに,$E[g^2]=1$と置いても一般性は失われない(以下の議論で$g$を$g/E[g^2]$に置き換えればよい).左辺は
    \begin{equation}
        \begin{split}
            E[g_1\cdots g_n]
            &=E[g^n] \\
            &=\int_{-\infty}^\infty x^n\frac{1}{\sqrt{2\pi}}e^{-\frac{x^2}{2}}dx \\
            &=(n-1)\int_{-\infty}^\infty x^{n-2}\frac{1}{\sqrt{2\pi}}e^{-\frac{x^2}{2}}dx \\
            &=\cdots \\
            &=(n-1)!!
        \end{split}
    \end{equation}
    であり,右辺は
    \begin{equation}
        \begin{split}
            \sum_{\pi\in\mathcal{P}_2(n)}E_\pi[g,\cdots,g]
            &=(n-1)!!E[g^2]^{\frac{n}{2}} \\
            &=(n-1)!!
        \end{split}
    \end{equation}
    となることから成り立つ.いま,補題において$\varphi(g_1,\ldots,g_n)=E[g_1\cdots g_n]-\sum_{\pi\in\mathcal{P}_2(n)}E_\pi[g_1,\ldots,g_n]$と置けば,$\varphi(g,\ldots,g)=0$は確認したので$\varphi(g_1,\ldots,g_n)=0$も成り立つ.よって示された.
\end{proof}

\subsection{Gaussian Unitary Ensemble and Wiegner's Theorem}

\begin{dfn}[Gaussian Unitary Ensemble]
    Gaussian Unitary Ensembleとは,行列$H=(h_{ij})_{i,j=1}^n$であって,すべての$i,j=1,\ldots,n$に対して$h_{ij}=\overline{h}_{ji}$を満たし,$\{h_{ij}\mid i\geq j\}$は独立同分布で$\mathcal{N}(0,1/n)$に従うようなもののことである.$H$のガウス測度(すなわち,$H$のすべての要素の同時分布)は
    \begin{equation}
        d\mu(H)=\frac{1}{Z}e^{-\frac{n}{2}\mathrm{tr}H^2}dH
    \end{equation}
    となる.ただし$Z=2^{\frac{n}{2}}\pi^{\frac{1}{2}n^2}$,$dH=\prod_{i\geq j}dh_{ij}$.
\end{dfn}

\begin{example}
    ランダム行列$X$は$\mathrm{GUE}(n)$であるとする.このとき,$E[X_{ij}X_{kl}]=\delta_{il}\delta_{jk}/n$が成り立つ.また,Wickの定理より$E[X_{i_1j_1}\cdots X_{i_kj_k}]=\sum_{\pi\in\mathcal{P}_2(k)}n^{-k/2}\delta_{\pi_{ij}}$
\end{example}

\begin{dfn}[互換の数]
    $k$次の対称群の元$\sigma\in S_k$に対して$|\sigma|$を互換の数として定める.
\end{dfn}

\begin{dfn}[サイクルの数]
    $k$次の対称群の元$\sigma\in S_k$に対して$\#(\sigma)$を,巡回置換の積に分解したときのサイクルの数として定める.
\end{dfn}

\begin{prop}
    \
    \begin{enumerate}
        \item $\sigma,\tau \in S_k$に対して,対応$(\sigma,\tau)\mapsto |\sigma\tau^{-1}|$を考えると,これは$S_k$の距離となる.すなわち,この対応を$d$で表すと以下の3つが成り立つ:
              \begin{itemize}
                  \item $d(\sigma,\tau)=0\Leftrightarrow \tau=\sigma$
                  \item $d(\sigma_1,\sigma_2)+d(\sigma_2,\sigma_3)\geq d(\sigma_1,\sigma_3)$
                  \item $d(\sigma_1,\sigma_2)=d(\sigma_2,\sigma_1)$
              \end{itemize}
        \item $\sigma\in S_k$について,$|\sigma|+\#(\sigma)=k$が成り立つ.
    \end{enumerate}
\end{prop}

\begin{example}
    以下のいずれの場合でも$|\sigma|+\#(\sigma)=k$が成立していることが確認できる.
    \begin{itemize}
        \item $\sigma=e\in S_k$を考えると,$|\sigma|=0$.$\sigma=(1)(2)\cdots(k)$より$\#(e)=k$.
        \item $\sigma=(1\,2)\in S_k$を考えると,$|\sigma|=1$,$\sigma=(1\,2)(3)\cdots(k)$より$\#(\sigma)=k-1$.
        \item $\sigma=(1\,2\,\cdots\,k)\in S_k$を考えると,$|\sigma|=|(1\,2)\cdots(k-1\,k)|=k-1$.$\#(\sigma)=1$.
    \end{itemize}
\end{example}

\begin{dfn}[normalized trace]
    $n$次元正方行列$X$に対して$\mathrm{tr}X=\frac{1}{n}\mathrm{Tr}X$と定める.
\end{dfn}

\begin{thm}
    $X$を$\mathrm{GUE}(n)$とする.このとき,$\gamma=(1\,2\,\cdots\,k)\in S_k$として以下が成り立つ:
    \begin{equation}
        E[\mathrm{tr}X^k]=\sum_{\pi\in\mathcal{P}_2(k)}n^{-1-\frac{k}{2}+\#(\gamma\pi)}
    \end{equation}
\end{thm}

\begin{proof}
    \begin{equation}
        E[\mathrm{tr}X^k]=\frac{1}{n}\sum_{1\leq i_1,i_2,\ldots,i_k\leq n}E[X_{i_1i_2}X_{i_2i_3}\cdots X_{i_ki_1}]
    \end{equation}
    ここで,Wickの定理より
    \begin{equation}
        \begin{split}
            E[X_{i_1i_2}X_{i_2i_3}\cdots X_{i_ki_1}]
            &=\sum_{\pi\in\mathcal{P}_2(k)}E_\pi[X_{i_1i_2},X_{i_2i_3},\ldots, X_{i_ki_1}] \\
            &=\sum_{\pi\in\mathcal{P}_2(k)}\prod_{(a,b)\in\pi}E[X_{i_ai_{a+1}}X_{i_bi_{b+1}}]
        \end{split}
    \end{equation}
    ただし$i_{k+1}=i_1$とする.ここで
    \begin{equation}
        E[X_{i_ai_{a+1}}X_{i_bi_{b+1}}]=\frac{1}{n}\delta_{i_ai_{a+1}}\delta_{i_bi_{b+1}}
    \end{equation}
    であることと,$(a,b)\in\pi$は$\pi(a)=b,\pi(b)=a$であるということなので
    \begin{equation}
        \prod_{(a,b)\in\pi}\delta_{i_ai_{a+1}}\delta_{i_bi_{b+1}}=\prod_{a=1}^k\delta_{i_ai_{\pi(a)+1}}
    \end{equation}
    となる.ここでshift permutation $\gamma\in S_k$(すなわち$\gamma=(1\,2\,\cdots\,k),\gamma(a)=a+1\mod k$)を定めると結局
    \begin{equation}
        \begin{split}
            E[\mathrm{tr}X^k]=\frac{1}{n}\frac{1}{n^{k/2}}\sum_{\pi\in\mathcal{P}_2(k)}\sum_{1\leq i_1,i_2,\ldots,i_k\leq n}\prod_{a=1}^k\delta_{i_ai_{\gamma\pi(a)}}
        \end{split}
    \end{equation}
    となる.ここで$\prod_{a=1}^k\delta_{i_ai_{\gamma\pi(a)}}\neq 0$であるのは$\gamma\pi$のサイクル上で一定となっていることである.したがって%どういうこと?
    \begin{equation}
        E[\mathrm{tr}X^k]=\frac{1}{n^{k/2+1}}\sum_{\pi\in\mathcal{P}_2(k)}n^{\#(\gamma\pi)}
    \end{equation}
\end{proof}

上の定理の別の表記を与える.

\begin{thm}
    $X$を$\mathrm{GUE}(n)$とする.このとき,$\gamma=(1\,2\,\cdots\,k)\in S_k$として
    \begin{equation}
        E[\mathrm{tr}X^k]=\sum_{\pi\in\mathcal{P}_2(k)}n^{|\gamma|-|\pi|-|\gamma\pi|}
    \end{equation}
    が成り立つ.
\end{thm}

\begin{proof}
    $|\gamma|=k-1$,$|\pi|=k/2$,$|\gamma\pi|=k-\#(\gamma\pi)$であることから確認できる.
\end{proof}

\begin{example}
    $X$が$\mathrm{GUE}(n)$であるとき,$E[\mathrm{tr}X^4]=2+\frac{1}{n}$
\end{example}

\begin{proof}
    定義とWickの定理から計算すれば
    \begin{equation}
        \begin{split}
            E[\mathrm{tr}X^4]
            &=\frac{1}{n^3}\sum_{i,j,k,l=1}^nE[X_{ij}X_{jk}X_{kl}X_{li}] \\
            &=\frac{1}{n^3}\sum_{i,j,k,l=1}^n(E_{\pi_1}+E_{\pi_2}+E_{\pi_3})[X_{ij}X_{jk}X_{kl}X_{li}]
        \end{split}
    \end{equation}

    各項について
    \begin{equation}
        \begin{split}
            \sum_{i,j,k,l=1}^n E_{\pi_2}[X_{ij},X_{jk},X_{kl},X_{li}]
            &=\sum_{i,j,k,l=1}^n E[X_{ij}X_{jk}]E[X_{kl}X_{li}] \\
            &=\sum_{i,j,k,l=1}^n \delta_{ik}\delta_{jj}\delta_{ki}\delta_{ll} \\
            &=n^3
        \end{split}
    \end{equation}
    \begin{equation}
        \begin{split}
            \sum_{i,j,k,l=1}^n E_{\pi_1}[X_{ij},X_{jk},X_{kl},X_{li}]
            &=\sum_{i,j,k,l=1}^n E[X_{ij}X_{kl}]E[X_{jk}X_{li}] \\
            &=\sum_{i,j,k,l=1}^n \delta_{il}\delta_{jk}\delta_{ji}\delta_{kl} \\
            &=n
        \end{split}
    \end{equation}
    \begin{equation}
        \begin{split}
            \sum_{i,j,k,l=1}^n E_{\pi_3}[X_{ij},X_{jk},X_{kl},X_{li}]
            &=\sum_{i,j,k,l=1}^n E[X_{ij}X_{li}]E[X_{jk}X_{kl}] \\
            &=\sum_{i,j,k,l=1}^n \delta_{ii}\delta_{jl}\delta_{jl}\delta_{kk} \\
            &=n^3
        \end{split}
    \end{equation}
    となることから従う.

    また,先の定理を使えば,とられる和の中身が以下の表のようになることからもわかる.
    \begin{table}[hbtp]
        \centering
        \begin{tabular}{cccc}
            \hline
            $\pi$          & $\gamma\pi$    & $\#(\gamma\pi)-3$ & 貢献度   \\
            \hline \hline
            $(1\,2)(3\,4)$ & $(1\,3)(2)(4)$ & $0$               & $n^0=1$  \\
            $(1\,3)(2\,4)$ & $(1\,4\,3\,2)$ & $-2$              & $n^{-2}$ \\
            $(1\,4)(2\,3)$ & $(1)(2\,4)(3)$ & $0$               & $n^0=1$  \\
            \hline
        \end{tabular}
    \end{table}
\end{proof}

\begin{dfn}[non-crossing]
    $\pi\in\mathcal{P}_2(m)$がnon-crossingとは,すべての$\pi$の要素$(i,k)$と$(j,l)$について,$i<j<k<l$とはならないことである.non-crossingな集合のことを
    \begin{equation}
        \mathcal{NC}_2(m)=\{\pi\in\mathcal{P}_2(m)\mid\pi\,\textrm{ is\, non-crossing}\}
    \end{equation}
    と表記する.
\end{dfn}

\begin{thm}[Biane's lemma]
    $\pi\in\mathcal{P}_2(k)\subset S_k$とし,$\gamma=(1\,2\,\cdots\,k)\in S_k$とする.
    \begin{enumerate}
        \item $\#(\gamma\pi)\leq\frac{k}{2}+1$.
        \item $\#(\gamma\pi)=\frac{k}{2}+1$であることは$\pi\in\mathcal{NC}_2(k)$であるための必要十分条件である.
    \end{enumerate}
    \begin{comment}
    別の書き方をすれば
    \begin{enumerate}
        \item $|\gamma|\leq |\pi|+|\gamma\pi|$.
        \item $|\gamma|=|\pi|+|\gamma\pi|$であることは,$\pi\in\mathcal{NC}_2(k)$であるための必要十分条件である.
    \end{enumerate}
    \end{comment}
\end{thm}

\begin{proof}
    $\tau=(i\,i+1)\in\pi$であったとすれば,$\gamma\pi(i+1)=i+1$,$\gamma\pi(i)=i+2$となる.したがって,$\gamma\pi$はサイクル$(i+1)$と$(\cdots\,i\,i+2\,\cdots)$を含む.このとき$i,i+1$を取り除く操作を行う.その場合の$\pi$と$\gamma$を改めて$\pi_2$と$\gamma_2$として考え,再び同様にして取り除く操作を繰り返すことを考える.

    もし$\pi\in\mathcal{NC}_2(k)$であれば,上の操作は$\pi_{k/2-1}=(1\,2)$,$\gamma_{k/2-1}=(1\,2)$,$\gamma_{k/2-1}\pi_{k/2-1}=(1)(2)$,$\#(\gamma_{k/2-1}\pi_{k/2-1})=2$となるまで繰り返すことができる.これは最初の状態から上の操作を$k/2-1$回繰り返すことで得られる状態である.よって$\#(\gamma\pi)=(k/2-1)+\#(\gamma_{k/2-1}\pi_{k/2-1})=k/2+1$となる.

    一方で,$\pi\notin\mathcal{NC}_2(k)$であれば,上の操作を繰り返し適用し,これ以上操作ができない状態に到達したとして,そのときの$\pi_i\gamma_i$を考えれば,$\pi_i\gamma_i(j)=j$となるような$j$が存在しない.すなわち,$\pi_i\gamma_i$に存在するサイクルは必ず2つ以上の要素によって構成されていることになる.したがって,始めの状態に戻して考えることで$\#(\gamma\pi)\leq k/2<k/2+1$であることがわかる.
\end{proof}


\begin{lem}
    \begin{equation}
        \#\mathcal{NC}_2(n)=
        \begin{cases}
            0                                      & (nは奇数) \\
            \mathrm{Cat}\mleft(\frac{n}{2}\mright) & (nは偶数)
        \end{cases}
    \end{equation}
    ただし$\mathrm{Cat}$はカタラン数であり,$\mathrm{Cat}(m)=\frac{(2m)!}{m!(m+1)!}$をみたす.
\end{lem}

\begin{proof}
    いま,$k\in[2(n+1)]$を任意に選んで,$k$と$2(n+1)$を結ぶことを考えると,$[k-1]$と$[2n+1]/[k]\cong [2n-(k-1)]$の2つの集合についてnon-crossing pairを考えればよいことになるので,$n$が奇数のときは$\#\mathcal{NC}_2(n)=0$であることに注意して,全ての$k$について和をとり
    \begin{equation}
        \begin{split}
            \#\mathcal{NC}_2(2(n+1))&=\sum_{k=1}^{2n+1}\#\mathcal{NC}_2(k-1)\#\mathcal{NC}_2(2n-(k-1)) \\
            &=\sum_{k=0}^{n}\#\mathcal{NC}_2(2k)\#\mathcal{NC}_2(2(n-k))
        \end{split}
    \end{equation}
    となる.ここで$c_{n}=\#\mathcal{NC}_2(2n)$とおくことで
    \begin{equation}
        c_{n+1}=\sum_{k=0}^n c_kc_{n-k}
    \end{equation}
    となる.この$c_n=\mathrm{Cat}(n)$である.実際,この数列の母関数$f(x)$を考えると
    \begin{equation}
        f(x)=\sum_{k=0}^\infty c_kx^k
    \end{equation}
    となるが,
    \begin{equation}
        \begin{split}
            f(x)^2&=\sum_{n=0}^\infty\sum_{k=0}^n c_kc_{n-k} x^n \\
            &=\sum_{n=0}^\infty c_{n+1}x^n \\
            &=\frac{1}{x}(f(x)-1)
        \end{split}
    \end{equation}
    であることより
    \begin{equation}
        f(x)^2x-f(x)+1=0
    \end{equation}
    を解いて(ただし$f(0)=c_0=1$で連続となるように符号をとる)
    \begin{equation}
        \begin{split}
            f(x)&=\frac{1-\sqrt{1-4x}}{2x} \\
            &=\sum_{n=1}^\infty \frac{(2n)!}{(n+1)!n!}x^n
        \end{split}
    \end{equation}
    係数を比較することで$c_n=\frac{(2n)!}{(n+1)!n!}$を得る.

\end{proof}

$\lim_{n\to\infty}E[\mathrm{tr} X^k]$の値を考えたい.このとき,$\#(\gamma\pi)\leq \frac{k}{2}+1$であることから,$\pi\in\mathcal{P}_2(k)$が$\#(\gamma\pi)= \frac{k}{2}+1$をみたす場合だけ考えれば十分.なお,幾何的には$\pi\in\mathcal{P}_2(k)$が先の命題で定めた距離$d$に関する測地線上にあることを意味する.

\begin{thm}
    $X$は$\mathrm{GUE}(n)$であるとする.このとき,以下が成り立つ:
    \begin{equation}
        \lim_{n\to\infty}E[\mathrm{tr} X^k]=\#\mathcal{NC}_2(k)
    \end{equation}
\end{thm}

\begin{dfn}[semicircular distribution]
    \begin{equation}
        ds(x)=\frac{1}{2\pi}\sqrt{4-x^2}\mathbbm{1}_{\{|x|\leq2\}}
    \end{equation}
\end{dfn}

\begin{proof}
    部分積分により示せる.
\end{proof}

\begin{example}[Haar測度]
    $U\in\mathrm{SU}(2)$をとると,$\mathrm{Tr}U$は$s$による分布をもつ.
\end{example}

\begin{lem}
    \begin{equation}
        \int x^k ds(x)=\begin{cases}
            0                         & (kは奇数) \\
            \mathrm{Cat}(\frac{k}{2}) & (kは偶数)
        \end{cases}
    \end{equation}
\end{lem}

以上より

\begin{thm}
    $X^{(n)}$を$\mathrm{GUE}(n)$とする.任意の実数係数多項式$p\in\mathrm{R}[x]$に対して
    \begin{equation}
        \lim_{n\to\infty}E[\mathrm{tr}p(X^{(n)})]=\int p(x)ds(x)
    \end{equation}
    が成り立つ.
\end{thm}

\begin{thm}
    任意の実数係数多項式$p\in\mathrm{R}[x]$に対して
    \begin{equation}
        \lim_{n\to\infty}E[\mathrm{tr}p(X^{(n)})]=\int p(x)ds(x)
    \end{equation}
    が成り立つ.
\end{thm}


\begin{dfn}
    エルミート行列$Z$の標準化された固有値計数測度を,$Z$の固有値を$\lambda_1(Z)\geq\cdots\geq\lambda_n(Z)$と表記して
    \begin{equation}
        \mu_Z=\frac{1}{n}\sum_{i=1}^n\delta_{\lambda_i(Z)}
    \end{equation}
    で定める.また
    \begin{equation}
        \mu_Z([a,b])=\frac{1}{n}\#\{i\mid\lambda_i(Z)\in[a,b]\}
    \end{equation}
    である.
\end{dfn}

\begin{lem}
    \begin{equation}
        \mathrm{tr}p(Z)=\int p(t)d\mu_Z(t)
    \end{equation}
\end{lem}

\begin{proof}
    \begin{equation}
        \begin{split}
            \mathrm{tr}p(Z)&=\frac{1}{n}\sum_{i=1}^np(\lambda_i(Z))\\
            &=\int_\mathbb{R}p(x)\frac{1}{n}\sum_{i=1}^n\delta_{\lambda_i(Z)}(x)
        \end{split}
    \end{equation}
\end{proof}

$Z$が決定論的なら$\mu_Z$は決定論的,ランダムならばランダム確率測度

$\mu_X$がランダム確率測度ならば特に$E\mu_{X}$がwell-defined.これは任意の連続関数$f$に対して
\begin{equation}
    E\int f(x)d\mu_{X}(x)=\int f(x)dE\mu_{X}(x)
\end{equation}
である.これを用いると任意の実数係数多項式$p$に対して
\begin{equation}
    \int p(t)dE\mu_{X}(t)\overset{n\to\infty}{\longrightarrow}\int p(t)ds(t)
\end{equation}
であることが証明できる.さらに,
\begin{thm}
    任意の有界連続関数$f$に対して
    \begin{equation}
        \int f(t)dE\mu_{X}(t)\overset{n\to\infty}{\longrightarrow}\int f(t)ds(t)
    \end{equation}
    が成り立つ.
\end{thm}
このことは
\begin{equation}
    E\mu_{X}\overset{\mathrm{weak}}{\longrightarrow} s
\end{equation}
であることを主張している.

\begin{thm}[Wiegner's theorem]
    $X^{(n)}$を$\mathrm{GUE}(n)$とする.このとき
    \begin{equation}
        E\mu_{X^{(n)}}\overset{n\to\infty}{\longrightarrow} s\quad(\textrm{in\, weak})
    \end{equation}
\end{thm}

\end{document}
